\documentclass[11pt,notitlepage,a4paper]{article}
%\usepackage[a4paper,hmargin=1.3in,vmargin=1.3in]{geometry}
\usepackage[
  bookmarks=true,       % show bookmarks bar?
  unicode=true,         % non-Latin characters in Acrobat’s bookmarks
  pdftoolbar=false,     % show Acrobat’s toolbar?
  pdfmenubar=false,     % show Acrobat’s menu?
  pdffitwindow=false,   % window fit to page when opened
  pdfstartview={FitH},  % fits the width of the page to the window
  pdftitle={BLAKE3},    % title
  pdfauthor={TODO},     % author
  pdfsubject={BLAKE3},  % subject of the document
  pdfnewwindow=true,    % links in new window
  colorlinks=true,      % false: boxed links; true: colored links
  linkcolor=orangered,  % color of internal links
  citecolor=orangered,   % color of links to bibliography
  filecolor=magenta,    % color of file links
  urlcolor=orangered,   % color of external links
]{hyperref}
\usepackage{url}
%\usepackage{fullpage}
\usepackage{a4wide}
\usepackage{amsmath,amssymb,cite,dsfont}

\usepackage{verbatim,footmisc}
\usepackage{array}

%\usepackage{euler}
%\usepackage{helvet}
%\renewcommand{\familydefault}{\sfdefault}
%\fontfamily{phv}\selectfont

\renewcommand{\familydefault}{\sfdefault}
\usepackage[scaled=1]{helvet}
\usepackage[helvet]{sfmath}
\everymath={\sf}

\usepackage[T1]{fontenc} % improves underscore rendering

\usepackage{textcomp}
\usepackage{microtype}
\usepackage{multirow}
\usepackage{booktabs}
\usepackage{datetime}
\usepackage{xcolor}
\usepackage{graphicx}
\usepackage{subfigure}

\usepackage[english]{babel}
\usepackage[autostyle,english=american]{csquotes}

\usepackage{xspace}
\usepackage{booktabs}
\usepackage{listings}
\usepackage{tikz}
\usepackage[shortcuts]{extdash} % unbreakable dashes, e.g. SHA\=/2
\usepackage[outputdir=build]{minted} % syntax highlighting using Pygments
\usepackage[title]{appendix} % add "Appendix" to each appendix section title
\usepackage[font=small,labelfont=bf]{caption}
\usetikzlibrary
{
  calc,
  positioning,
  shapes,
  shapes.geometric,
  arrows
}
\usepackage{pgfplots}
\pgfplotsset{
    compat=newest,
    % As per https://tex.stackexchange.com/a/350514/201018, this lets us force
    % one graph line to the top, so that it's always drawn overlapping other
    % lines. Used in img/avx512.tikz.
    layers/my layer set/.define layer set={main, foreground}{},
    set layers=my layer set,
}

\definecolor{darkred}{RGB}{170,0,0}
\definecolor{darkblue}{RGB}{0,0,170}
\definecolor{darkgreen}{RGB}{0,170,0}
\definecolor{orangered}{RGB}{255,69,0}
\definecolor{bluecompl}{RGB}{0,186,255}

\newdateformat{mydate}{\THEYEAR.\twodigit\THEMONTH.\twodigit\THEDAY}

\newcommand{\GG}{\mathsf{G}}
\newcommand{\IV}{\text{IV}}
\newcommand{\cpress}{\text{\textbf{compress}}}
\newcommand{\lto}{\leftarrow}
\newcommand{\BB}{\mathsf{B2}}

\newcommand{\flag}[1]{\texttt{\detokenize{#1}}\xspace}

\newcommand{\alert}[1]{\textcolor{red}{#1}}

\newcommand{\mytitle}{BLAKE3}

\title{\mytitle}


\begin{document}

\fontfamily{phv}\selectfont
\pagestyle{plain}

{\let\thefootnote\relax\footnotetext{Version \texttt{\pdfdate}.}}

\begin{center}
{\Huge \bf \mytitle}

\medskip

{\Large \bf  one function, fast everywhere}

\medskip

Jack O'Connor (\texttt{@oconnor663}) \\
Jean-Philippe Aumasson (\texttt{@veorq}) \\
Samuel Neves (\texttt{@sevenps}) \\
Zooko Wilcox-O'Hearn (\texttt{@zooko}) \\

\medskip

{\large \url{https://blake3.io}}

\end{center}


\medskip

\begin{center}
  \begin{minipage}{0.92\linewidth}

      We present BLAKE3, an evolution of the BLAKE2 cryptographic hash that is
      both faster and also more consistently fast across different platforms
      and input sizes. BLAKE3 supports an unbounded degree of parallelism,
      using a tree structure that scales up to any number of SIMD lanes and CPU
      cores. On Intel Cascade Lake-SP, peak single-threaded throughput is
      $4\times$ that of BLAKE2b, $8\times$ that of SHA\=/512, and $12\times$
      that of SHA\=/256, and it can scale further using multiple threads.
      BLAKE3 is also efficient on smaller architectures: throughput on a 32-bit
      ARM1176 core is $1.3\times$ that of SHA\=/256 and $3\times$ that of
      BLAKE2b and SHA\=/512. Unlike BLAKE2 and SHA\=/2, with different variants
      better suited for different platforms, BLAKE3 is a single algorithm with
      no variants. It provides a simplified API supporting all the use cases of
      BLAKE2, including keying and extendable output. The tree structure also
      supports new use cases, such as verified streaming and incremental
      updates.

   \end{minipage}
\end{center}

\newpage

\tableofcontents

\newpage

\section{Introduction}\label{sec:intro}

Since its announcement in 2012, BLAKE2~\cite{DBLP:conf/acns/AumassonNWW13} has
seen widespread adoption, in large part because of its superior performance in
software. BLAKE2b and BLAKE2s are included in OpenSSL and in the Python and Go
standard libraries. BLAKE2b is also included as the \texttt{b2sum} utility in
GNU Coreutils, as the \texttt{generichash} API in Libsodium, and as the
underlying hash function for Argon2~\cite{DBLP:conf/eurosp/BiryukovDK16}, the
winner of the Password Hashing Competition in 2015.

The biggest drawback of BLAKE2 has been its large number of incompatible
variants. The performance tradeoffs between different variants are subtle, and
library support is uneven. BLAKE2b is the most widely supported, but it is not
the fastest on most platforms. BLAKE2bp and BLAKE2sp are more than twice as
fast on modern x86 processors, but they are sparsely supported and rarely
adopted.

BLAKE3 eliminates this drawback. It is a single algorithm with no variants,
designed for consistent high performance in software on all platforms. The
biggest changes from BLAKE2 to BLAKE3 are:

\begin{itemize}
    \item An \textbf{internal tree structure}, which supports an unbounded
        degree of parallelism. This is the largest performance improvement.
    \item A compression function with \textbf{fewer rounds}. This is also a
        performance improvement.
    \item \textbf{Three domain-separated modes}: \flag{hash(input)},
        \flag{keyed_hash(key, input)}, and
        % Paragraph layout overruns the margin without this linebreak. But why?
        \linebreak
        \flag{derive_key(context, key_material)}. This three-part API replaces
        the BLAKE2 parameter block.
    \item \textbf{Zero-cost keying}, using the space formerly occupied by the
        parameter block.
    \item Built-in support for \textbf{extendable output}. Like BLAKE2X, but
        unlike SHA\=/3 or HKDF, extended output is parallelizable and seekable.
\end{itemize}

BLAKE3 splits its input into 1~KiB chunks and arranges those chunks as the
leaves of a binary tree. This tree structure means that there is no limit to
the parallelism that BLAKE3 can exploit, given enough
input~\cite{DBLP:journals/tosc/AtighehchiB17,DBLP:journals/tc/AtighehchiR17}.
The direct benefit of this parallelism is very high throughput on platforms
with SIMD support, including all modern x86 processors. Another benefit of
hashing chunks in parallel is that the implementation can use SIMD vectors of
any width, regardless of the word size of the compression function. That leaves
us free to select a compression function that is efficient on smaller
architectures, without sacrificing peak throughput on x86\=/64.

The BLAKE3 compression function is closely based on that of BLAKE2s. BLAKE3 has
the same 128-bit security level and 256-bit default output size. The internal
mixing function is the same, with a simplified message schedule derived from a
single repeated permutation. Based on existing cryptanalysis of BLAKE and
BLAKE2, BLAKE3 reduces the number of rounds in the compression function from 10
to 7 (see~\cite{TMC} for a detailed rationale). BLAKE3 also changes the setup
and finalization steps of the compression function to support the internal tree
structure, more efficient keying, and extendable output.

\section{Specification}\label{sec:specification}

\subsection{Tree Structure}\label{sec:tree}

BLAKE3 splits its input into chunks of up to 1024 bytes and arranges those
chunks as the leaves of a binary tree. The last chunk may be shorter, but not
empty, unless the entire input is empty. If there is only one chunk, that chunk
is the root node and only node of the tree. Otherwise, the chunks are assembled
with parent nodes, each parent node having exactly two children. The
structure of the tree is determined by two rules:
\begin{enumerate}
    \item Left subtrees are full. Each left subtree is a complete binary tree,
        with all its chunks at the same depth, and a number of chunks that is a
        power of 2.
    \item Left subtrees are big. Each left subtree contains a number of chunks
        greater than or equal to the number of chunks in its sibling right
        subtree.
\end{enumerate}
In other words, given a message of $n > 1024$ bytes, the left subtree consists
of the first $$2^{10 + \left\lfloor \log_2 \left(\left\lfloor \frac{n-1}{1024}
\right\rfloor\right) \right\rfloor}$$ bytes, and the right subtree consists of
the remainder. For example, trees from 1 to 4 chunks have the structure shown
in Figure~\ref{fig:fourchunks}.

\begin{figure}[h]
\centering
\begin{tikzpicture}

\node[draw,rectangle] (chunk1) at (0, 0) {$c_0$};

%%

\node[draw,rectangle,right=1cm of chunk1] (chunk2) {$c_0$};
\node[draw,rectangle,right=0.5cm of chunk2] (chunk3) {$c_1$};
\node[draw,rectangle,above=1cm of chunk2] (parent1) at ($(chunk2)!0.5!(chunk3)$) {\phantom{$p_0$}};
\draw[-latex] (chunk2) -- (parent1);
\draw[-latex] (chunk3) -- (parent1);

%%

\node[draw,rectangle,right=1cm of chunk3] (chunk2) {$c_0$};
\node[draw,rectangle,right=0.5cm of chunk2] (chunk3) {$c_1$};
\node[draw,rectangle,above=1cm of chunk2] (parent1) at ($(chunk2)!0.5!(chunk3)$) {\phantom{$p_0$}};
\node[draw,rectangle,right=0.5cm of parent1] (chunk4) {$c_2$};
\node[draw,rectangle,above=1cm of chunk4] (parent2) at ($(parent1)!0.5!(chunk4)$) {\phantom{$p_0$}};
\draw[-latex] (chunk2) -- (parent1);
\draw[-latex] (chunk3) -- (parent1);
\draw[-latex] (parent1) -- (parent2);
\draw[-latex] (chunk4) -- (parent2);

%%

\node[draw,rectangle,below right=1cm of chunk4] (chunk2) {$c_0$};
\node[draw,rectangle,right=0.5cm of chunk2] (chunk3) {$c_1$};
\node[draw,rectangle,above=1cm of chunk2] (parent1) at ($(chunk2)!0.5!(chunk3)$) {\phantom{$p_0$}};
\node[draw,rectangle,right=0.5cm of chunk3] (chunk4) {$c_2$};
\node[draw,rectangle,right=0.5cm of chunk4] (chunk5) {$c_3$};
\node[draw,rectangle,above=1cm of chunk4] (parent2) at ($(chunk4)!0.5!(chunk5)$) {\phantom{$p_0$}};
\node[draw,rectangle,above=1cm of parent2] (parent3) at ($(parent1)!0.5!(parent2)$) {\phantom{$p_0$}};
\draw[-latex] (chunk2) -- (parent1);
\draw[-latex] (chunk3) -- (parent1);
\draw[-latex] (chunk4) -- (parent2);
\draw[-latex] (chunk5) -- (parent2);
\draw[-latex] (parent1) -- (parent3);
\draw[-latex] (parent2) -- (parent3);

\end{tikzpicture} 
\caption{Example tree structures, from 1 to 4 chunks.}
\label{fig:fourchunks}
\end{figure}

The compression function is used to derive a chaining value from each chunk and
parent node. The chaining value of the root node, encoded as 32 bytes in
little-endian order, is the default-length BLAKE3 hash of the input. BLAKE3
supports input of any byte length $0 \leq \ell < 2^{64}$.

\subsection{Compression Function}\label{sec:compression}

The compression function takes an 8-word chaining value, a 16-word message
block, and 4 other parameter words, and it returns a new 16-word value. A word
is 32 bits. The inputs are:

\begin{itemize}
    \item The input chaining value, $h_{0} \ldots h_{7}$.
    \item The message block, $m_{0} \ldots m_{15}$.
    \item A 64-bit counter, $t=t_{0},t_{1}$, with $t_{0}$ the lower order word
        and $t_{1}$ the higher order word.
    \item The number of input bytes in the block, $b$.
    \item A set of domain separation bit flags, $d$.
\end{itemize}

The compression function initializes its 16-word internal state $v_{0} \ldots
v_{15}$ as follows: 

\begin{equation*}
\begin{pmatrix}
v_{0} & v_{1} & v_{2} & v_{3} \\
v_{4} & v_{5} & v_{6} & v_{7} \\
v_{8} & v_{9} & v_{10} & v_{11} \\
v_{12} & v_{13} & v_{14} & v_{15} \\
\end{pmatrix}
\leftarrow
\begin{pmatrix}
h_{0} & h_{1} & h_{2} & h_{3} \\
h_{4} & h_{5} & h_{6} & h_{7} \\
\IV_{0} & \IV_{1} & \IV_{2} & \IV_{3} \\
t_{0} & t_{1} & b & d 
\end{pmatrix}
\end{equation*}

The $\IV_{0} \ldots \IV_{7}$ constants are the same as in
BLAKE2s, and they are reproduced in Appendix~\ref{sec:ivconstants}.

The compression function applies a 7-round keyed permutation $v' = E(m, v)$ to
the state $v_0 \dots v_{15}$, keyed by the message $m_0 \dots m_{15}$. The
keyed permutation here is identical to that of BLAKE2s, and is
reproduced in Appendix~\ref{sec:roundfn}.

The output of the compression function $h_{0}' \ldots h_{15}'$ is
defined as:
\begin{align*}
h_{0}'  \ & \leftarrow \ v'_{0} \oplus  v'_{8} &
h_{8}'  \ & \leftarrow \ v'_{8} \oplus  h_{0} \\
h_{1}'  \ & \leftarrow \ v'_{1} \oplus  v'_{9} &
h_{9}'  \ & \leftarrow \ v'_{9} \oplus  h_{1} \\
h_{2}'  \ & \leftarrow \ v'_{2} \oplus  v'_{10} &
h_{10}' \ & \leftarrow \ v'_{10} \oplus  h_{2} \\
h_{3}'  \ & \leftarrow \ v'_{3} \oplus  v'_{11} &
h_{11}' \ & \leftarrow \ v'_{11} \oplus  h_{3} \\
h_{4}'  \ & \leftarrow \ v'_{4} \oplus  v'_{12} &
h_{12}' \ & \leftarrow \ v'_{12} \oplus  h_{4} \\
h_{5}'  \ & \leftarrow \ v'_{5} \oplus  v'_{13} &
h_{13}' \ & \leftarrow \ v'_{13} \oplus  h_{5} \\
h_{6}'  \ & \leftarrow \ v'_{6} \oplus  v'_{14} &
h_{14}' \ & \leftarrow \ v'_{14} \oplus  h_{6} \\
h_{7}'  \ & \leftarrow \ v'_{7} \oplus  v'_{15} &
h_{15}' \ & \leftarrow \ v'_{15} \oplus  h_{7}\,.
\end{align*}
If we define $v_l$ (resp. $v'_l$) and $v_h$ (resp. $v'_h$) as the first and last 8-words of the input (resp. output) of $E(m, v)$, the compression function may be written as
\[
  \begin{pmatrix}
    1 & 1 \\
    0 & 1
  \end{pmatrix}\cdot%
  \begin{pmatrix}
  v'_l \\ v'_h
  \end{pmatrix} + 
  \begin{pmatrix}
   0 \\ v_l
  \end{pmatrix}\,.
\]
The output of the compression function is most often truncated to produce 256-bit chaining values. %We defer to analysis of the compression function to \S\ref{sec:security}.

%\subsubsection{Flags}
The compression function input $d$ is a bitfield, with each individual flag consisting of a power of 2. The value of $d$ is the sum of all the flags defined for a given
compression. Their names and values are given in Table~\ref{tab:flags}.
\begin{table}
  \centering
  \caption{Admissible values for input $d$ in the BLAKE3 compression function.}%
  \label{tab:flags}
  \begin{tabular}{cc}
    \toprule
    Flag name                  & Value \\ \midrule
    \flag{CHUNK_START}         & $2^0$ \\
    \flag{CHUNK_END}           & $2^1$ \\
    \flag{PARENT}              & $2^2$ \\
    \flag{ROOT}                & $2^3$ \\
    \flag{KEYED_HASH}          & $2^4$ \\
    \flag{DERIVE_KEY_CONTEXT}  & $2^5$ \\
    \flag{DERIVE_KEY_MATERIAL} & $2^6$ \\
    \bottomrule
  \end{tabular}
\end{table}

\subsection{Modes}\label{sec:modes}

BLAKE3 defines three domain-separated modes: \flag{hash}, \flag{keyed_hash},
and \flag{derive_key}. The first two modes differ from each other in their key
words $k_{0} \ldots k_{7}$ and in the additional flags they set for every call
to the compression function:

\begin{itemize}
  \item \flag{hash}: $k_0 \ldots k_7$ are the constants $\IV_0 \ldots \IV_7$, and no additional flags are set. 
  \item \flag{keyed_hash}: $k_0 \ldots k_7$ are parsed in little-endian order from the 32-byte key given by the caller, and the \flag{KEYED_HASH} flag is set for every compression.
\end{itemize}

The third mode, \flag{derive_key}, has two stages. First the context string is
hashed, with $k_0 \ldots k_7$ set to the constants $\IV_0 \ldots \IV_7$, and
the \flag{DERIVE_KEY_CONTEXT} flag set for every compression. Then the key
material is hashed, with $k_0 \ldots k_7$ set to the first 8 output words of
the first stage, and the \flag{DERIVE_KEY_MATERIAL} flag set for every
compression.

\subsection{Chunk Chaining Values}\label{sec:chunk}

Processing a chunk is structurally similar to the sequential hashing mode of BLAKE2. Each chunk of up to 1024 bytes is split into blocks of up to 64 bytes.
The last block of the last chunk may be shorter, but not empty, unless the
entire input is empty. If necessary, the last block is padded with zeros to be 64
bytes. 

Each block is parsed in little-endian order into message words $m_{0}
\ldots m_{15}$ and compressed. The input chaining value $h_{0} \ldots h_{7}$
for the first block of each chunk is comprised of the key words $k_{0} \ldots k_{7}$. The
input chaining value for subsequent blocks in each chunk is the output of the truncated
compression function for the previous block. 

The remaining compression function parameters are handled as follows (see also Figure~\ref{fig:chunk} for an example):
\begin{itemize}
\item The counter $t$ for each block is the chunk index, i.e., $0$ for all
blocks in the first chunk, $1$ for all blocks in the second chunk, and so on.
\item The block length $b$ is the
number of input bytes in each block, i.e., $64$ for all blocks except the last block of
the last chunk, which may be short. 
\item The first block of each chunk sets the
\flag{CHUNK_START} flag (cf. Table~\ref{tab:flags}), and the last block of each chunk sets the
\flag{CHUNK_END} flag. If a chunk contains only one block, that block sets
both \flag{CHUNK_START} and \flag{CHUNK_END}. If a chunk is the root of
its tree, the last block of that chunk also sets the \flag{ROOT} flag.
\end{itemize}

The output of the truncated compression function for the last block in a chunk
is the chaining value of that chunk.

\begin{figure}
\centering
\begin{tikzpicture}[
comp/.style={draw, trapezium, trapezium left angle=0, trapezium right angle=45, shape border rotate=270, minimum height=0.5cm, minimum width=1.5cm, trapezium stretches=false,anchor=north},
fullcomp/.style={draw, rectangle, minimum height=1.5cm, minimum width=0.75cm, trapezium stretches=false,anchor=north},
]

\begin{scope}[xshift=0cm]
\node[comp] (f0) {$F$};
\node[above=0.5cm of f0] (m0) {$m_0$};
\coordinate (tmp) at ($(f0.north west) + (-0.5cm, -0.25cm)$);
\node[below = 0.00cm of tmp,scale=0.50,anchor=east] (iv) {$\mathrm{IV}$};
\node[below = 0.33cm of tmp,scale=0.50,anchor=east] (t0) {$t=0$};
\node[below = 0.66cm of tmp,scale=0.50,anchor=east] (b0) {$b=64$};
\node[below = 0.99cm of tmp,scale=0.50,anchor=east] (d0) {$d=2^0$};

\draw[-latex] (m0) -- (f0);
\draw[-latex] (iv) -- (iv -| f0.west);
\draw[-latex] (t0) -- (t0 -| f0.west);
\draw[-latex] (b0) -- (b0 -| f0.west);
\draw[-latex] (d0) -- (d0 -| f0.west);
\end{scope}

\begin{scope}[xshift=3cm]
\node[comp] (f1) {$F$};
\node[above=0.5cm of f1] (m1) {$m_1$};
\coordinate (tmp) at ($(f1.north west) + (-0.5cm, -0.25cm)$);

\node[below = 0.33cm of tmp,scale=0.5,anchor=east] (t1) {$t=0$};
\node[below = 0.66cm of tmp,scale=0.5,anchor=east] (b1) {$b=64$};
\node[below = 0.99cm of tmp,scale=0.5,anchor=east] (d1) {$d=0$};

\draw[-latex] (m1) -- (f1);
\draw[-latex] (t1) -- (t1 -| f1.west);
\draw[-latex] (b1) -- (b1 -| f1.west);
\draw[-latex] (d1) -- (d1 -| f1.west);
\end{scope}

\begin{scope}[xshift=6cm]
\node[fullcomp] (f2) {$F$};
\node[above=0.5cm of f2] (m2) {$m_2$};
\coordinate (tmp) at ($(f2.north west) + (-0.5cm, -0.25cm)$);

\node[below = 0.33cm of tmp,scale=0.5,anchor=east] (t2) {$t=0$};
\node[below = 0.66cm of tmp,scale=0.5,anchor=east] (b2) {$b=53$};
\node[below = 0.99cm of tmp,scale=0.5,anchor=east] (d2) {$d=2^1+2^3$};

\draw[-latex] (m2) -- (f2);
\draw[-latex] (t2) -- (t2 -| f2.west);
\draw[-latex] (b2) -- (b2 -| f2.west);
\draw[-latex] (d2) -- (d2 -| f2.west);
\end{scope}

\begin{scope}[xshift=6cm,yshift=-3cm]
\node[fullcomp] (f3) {$F$};
\node[above=0.5cm of f3] (m3) {$m_2$};
\coordinate (tmp) at ($(f3.north west) + (-0.5cm, -0.25cm)$);

\node[below = 0.33cm of tmp,scale=0.5,anchor=east] (t3) {$t=1$};
\node[below = 0.66cm of tmp,scale=0.5,anchor=east] (b3) {$b=53$};
\node[below = 0.99cm of tmp,scale=0.5,anchor=east] (d3) {$d=2^1+2^3$};

\draw[-latex] (m3) -- (f3);
\draw[-latex] (t3) -- (t3 -| f3.west);
\draw[-latex] (b3) -- (b3 -| f3.west);
\draw[-latex] (d3) -- (d3 -| f3.west);
\end{scope}

\coordinate (tmp) at (f1.west |- iv);
\coordinate (tmp2) at ($(f0)!0.40!(f1)$);
\draw[-latex] (f0) -- (f0 -| tmp2) -- (tmp2 |- tmp) -- (tmp);
\coordinate (tmp) at (f2.west |- iv);
\coordinate (tmp2) at ($(f1)!0.40!(f2)$);
\draw[-latex] (f1) -- (f1 -| tmp2) -- (tmp2 |- tmp) -- (tmp);

\coordinate (tmp) at ($(f3.north west) + (-0cm, -0.25cm)$);
\coordinate (tmp2) at ($(f1.east) + (0.5cm, 0)$);
\draw[-latex] (f1.east) -- (tmp2) -- (tmp2 |- tmp) -- (tmp);

\node[right=0.5cm of f2] (out) {$h_0$};
\draw[-latex] (f2) -- (out);
\node[right=0.5cm of f3] (out) {$h_1$};
\draw[-latex] (f3) -- (out);

\end{tikzpicture}
\caption{Example of compression function inputs when hashing a 181-byte input $(m_0, m_1, m_2)$ into a 128-byte output $(h_0, h_1)$. Trapezia indicate that the compression function output is truncated to 256 bits.}\label{fig:chunk}
\end{figure}

\subsection{Parent Node Chaining Values}\label{sec:parent}

Each parent node has exactly two children, each either a chunk or another
parent node. The chaining value of each parent node is given by a single call
to the compression function. The input chaining value $h_{0} \ldots h_{7}$ is
the key words $k_{0} \ldots k_{7}$. The message words $m_{0} \ldots m_{7}$ are
the chaining value of the left child, and the message words $m_{8} \ldots
m_{15}$ are the chaining value of the right child. The counter $t$ for parent
nodes is always 0. The number of bytes $b$ for parent nodes is always 64.
Parent nodes set the \flag{PARENT} flag. If a parent node is the root of the
tree, it also sets the \flag{ROOT} flag. The output of the truncated
compression function is the chaining value of the parent node.

\subsection{Extendable Output}\label{sec:extendable}

BLAKE3 can produce outputs of any byte length $0 \leq \ell < 2^{64}$. 
This is done by repeating the root compression---that is, the very last 
call to the compression function, which sets the \flag{ROOT} flag---with 
incrementing values of the counter $t$. The results of these repeated root 
compressions are then concatenated to form the output.

When building the output, BLAKE3 uses the full output of the
compression function (cf. \S\ref{sec:compression}). Each 16-word output is
encoded as 64 bytes in little-endian order.

Observe that based on \S\ref{sec:chunk} and \S\ref{sec:parent} above, the first
root compression always uses the counter value $t = 0$. That is either because
it is the last block of the only chunk, which has a chunk index of $0$, or
because it is a parent node. After the first root compression, as long as more
output bytes are needed, $t$ is incremented by 1, and the root compression is
repeated on otherwise the same inputs. If the target output length is not a
multiple of 64, the final compression output is truncated. See
Figure~\ref{fig:chunk} for an example.

Because the repeated root compressions differ only in the value of $t$, the
implementation can execute any number of them in parallel. The caller can also
adjust $t$ to seek to any point in the output stream.

Note that in contrast to BLAKE2 and BLAKE2X, BLAKE3 does not domain separate
outputs of different lengths. Shorter outputs are prefixes of longer ones. The
caller can extract as much output as needed, without knowing the final length
in advance.

\section{Performance}\label{sec:performance}

Several factors contribute to the performance of BLAKE3, depending on the
platform and the size of the input:

\begin{itemize}
    \item The implementation can compress groups of chunks in parallel using
        SIMD (cf.~\S\ref{sec:simd}). This is visible at 4~KiB and above in
        Figure~\ref{fig:avx512}.
    \item The implementation can use multiple threads
        (cf.~\S\ref{sec:multithreading}). This is the subject of
        Figure~\ref{fig:threads}. Note that Figures~\ref{fig:avx512}
        and~\ref{fig:rpizero} are single-threaded.
    \item The compression function performs fewer rounds than in BLAKE2s. This
        improves performance across the board, but it is especially visible in
        the difference between BLAKE3 and BLAKE2s in Figure~\ref{fig:rpizero}.
    \item The compression function uses 32-bit words (cf.
        \S\ref{sec:wordsize}), which perform well on smaller architectures.
        This is the subject of Figure~\ref{fig:rpizero}.
    \item The 64-byte block size is relatively small, which helps performance
        for very short inputs. This is visible at 64 bytes (the left edge) in
        Figures~\ref{fig:avx512} and~\ref{fig:rpizero}.
\end{itemize}

\begin{figure}[h]
\centering
\begin{tikzpicture}
\begin{axis}[
    width=0.9\textwidth,
	height=10cm,
	legend style={at={(0.98,0.98)}, anchor=north east, font=\small, cells={anchor=west}},
	axis x line=bottom,
	axis y line=left,
	cycle multi list={
		linestyles\nextlist
		blue,red,orange,green,cyan
	},
	%no markers,
	mark options={scale=0.5},
	every axis plot/.append style={thick},
    ymin=0,
    ymax=10,
    xmin=64,
    xmax=1048576,
	ylabel=Throughput (cpb),
	xlabel=Input Length,
	xmode=log,
	log basis x=2,
	y tick label style={%
		/pgf/number format/.cd,%
		scaled x ticks = false,%
		set thousands separator={},%
		fixed%
	},
	xticklabels={64 B,128 B,256 B,512 B,1 KiB,2 KiB,4 KiB,8 KiB,16 KiB,32 KiB,64 KiB,128 KiB,256 KiB,512 KiB,1 MiB},
	xtick={64, 128, 256, 512, 1024, 2048, 4096, 8192, 16384, 32768, 65536, 131072, 262144, 524288, 1048576},
	ytick={0, 0.49, 1, 2, 3, 4, 5, 6, 7, 8, 9, 10},
	x tick label style={rotate=90,anchor=east}
]
\legend{BLAKE3, BLAKE2b, BLAKE2s, BLAKE2bp, BLAKE2sp, SHA-512, SHA-256, KangarooTwelve}
% Use "on layer=foreground" to force this curve to the front, because it tends
% to get hidden otherwise. See https://tex.stackexchange.com/a/350514/201018.
% blake3
\addplot [blue, on layer=foreground] coordinates {
(64, 3.658508222516906)
(80, 5.042939438604006)
(96, 4.20501753260731)
(112, 3.604823721336843)
(128, 3.1541052274366073)
(160, 3.6261718243907386)
(192, 3.022259246888865)
(224, 3.373533782527106)
(256, 2.951840512537496)
(320, 2.9124027074582095)
(384, 2.8847883590728878)
(448, 2.8656722816881093)
(512, 2.8506139421909067)
(640, 2.8301915434608906)
(768, 2.8282584043242878)
(896, 2.8203833015004873)
(1024, 2.812190442053065)
(1280, 2.9599000038254983)
(1536, 2.9245731425462975)
(1792, 2.9056820817849727)
(2048, 1.5032532525891293)
(2560, 1.850346400499486)
(3072, 1.9704599114501027)
(3584, 2.0875528101599756)
(4096, 0.8575157842610952)
(5120, 1.2675568395443033)
(6144, 1.082704568953793)
(7168, 1.313276600124271)
(8192, 0.6855543966553985)
(10240, 0.8627008747934303)
(12288, 0.7487386986490294)
(14336, 0.847413796498034)
(16384, 0.5135112197437373)
(20480, 0.5875867823688973)
(24576, 0.5796160785266639)
(28672, 0.6153541621153547)
(32768, 0.500668233794207)
(40960, 0.5479009954863138)
(49152, 0.49867285171302256)
(57344, 0.5379426790470052)
(65536, 0.4958834687736147)
(81920, 0.4929059563640973)
(98304, 0.49159636411016455)
(114688, 0.49008409886098225)
(131072, 0.4901261665251141)
(163840, 0.49075199079910015)
(196608, 0.4885753163438209)
(229376, 0.48788676012301135)
(262144, 0.4874210961565795)
(327680, 0.48871093143365435)
(393216, 0.4868601510048053)
(458752, 0.48699527505844203)
(524288, 0.4878270422409957)
(655360, 0.4872712138838195)
(786432, 0.4881389217466905)
(917504, 0.4909604014689926)
(1048576, 0.496897051692696)
};
% blake2b
\addplot coordinates {
(64, 7.230231921292101)
(80, 5.760042853942316)
(96, 4.79687063798512)
(112, 4.120513230279892)
(128, 3.5900665266965883)
(160, 5.075123480312184)
(192, 4.21923472347255)
(224, 3.6129727385485753)
(256, 3.104077840586492)
(320, 3.6005050981097404)
(384, 2.977818833318164)
(448, 3.3419869657926022)
(512, 2.9102437543577877)
(640, 2.8663737433066174)
(768, 2.8416982456513176)
(896, 2.8079393546285516)
(1024, 2.8093973667511896)
(1280, 2.78960908278112)
(1536, 2.7759698276004863)
(1792, 2.7649357954058167)
(2048, 2.7595033378168896)
(2560, 2.7487392202636225)
(3072, 2.7418929494217767)
(3584, 2.7370945483896683)
(4096, 2.733765711542996)
(5120, 2.7120339900298913)
(6144, 2.725093667088431)
(7168, 2.7226906449576465)
(8192, 2.7209735398133423)
(10240, 2.7183552177461023)
(12288, 2.7165929620847096)
(14336, 2.715492213432871)
(16384, 2.714365793645063)
(20480, 2.712911176547196)
(24576, 2.7119482896021645)
(28672, 2.7116818322767204)
(32768, 2.7117351075773293)
(40960, 2.7111510598420208)
(49152, 2.711112289901746)
(57344, 2.6938695289131687)
(65536, 2.7100636157232287)
(81920, 2.7097568715285636)
(98304, 2.7012145666805765)
(114688, 2.6933522952955413)
(131072, 2.7029009897171834)
(163840, 2.699114179584832)
(196608, 2.6937763782604756)
(229376, 2.694683441368914)
(262144, 2.6943508318436975)
(327680, 2.691981870321751)
(393216, 2.692076753616615)
(458752, 2.6919445792792525)
(524288, 2.7072210984577434)
(655360, 2.692217529935007)
(786432, 2.8337715819358684)
(917504, 2.8157031589375903)
(1048576, 2.800198658459363)
};
% blake2s
\addplot coordinates {
(64, 5.277547125955014)
(80, 7.977540332001018)
(96, 6.379504974208327)
(112, 5.4689466203214545)
(128, 4.704607105380211)
(160, 5.536752684758274)
(192, 4.587987769581591)
(224, 5.130073421641601)
(256, 4.468993928312757)
(320, 4.429805325295102)
(384, 4.395787120788447)
(448, 4.391817750532874)
(512, 4.359107464371288)
(640, 4.333060937063378)
(768, 4.318990493449834)
(896, 4.305844211268713)
(1024, 4.299129916152097)
(1280, 4.286680577431882)
(1536, 4.2808410451704875)
(1792, 4.275192618485819)
(2048, 4.2694707078194)
(2560, 4.26475685658687)
(3072, 4.260631907837534)
(3584, 4.258832616778033)
(4096, 4.256682331836171)
(5120, 4.253054724772646)
(6144, 4.251411577294915)
(7168, 4.249990840100307)
(8192, 4.249229305882616)
(10240, 4.248376792699727)
(12288, 4.2476078642433635)
(14336, 4.246058512735586)
(16384, 4.246536067001818)
(20480, 4.2454059418269)
(24576, 4.244613851174787)
(28672, 4.245466706077716)
(32768, 4.244032336192421)
(40960, 4.241797472020698)
(49152, 4.2417395898875245)
(57344, 4.242233409337849)
(65536, 4.241915120744999)
(81920, 4.240721904766853)
(98304, 4.241098100773002)
(114688, 4.240764560428244)
(131072, 4.240740301696769)
(163840, 4.240335378718499)
(196608, 4.240167309431412)
(229376, 4.239919721374724)
(262144, 4.23992906472629)
(327680, 4.239512750394767)
(393216, 4.239715897684279)
(458752, 4.240097588252411)
(524288, 4.240693799373018)
(655360, 4.240666406626114)
(786432, 4.243268002904252)
(917504, 4.24563880884171)
(1048576, 4.245794605107433)
};
% blake2bp
\addplot coordinates {
(64, 27.583448172105403)
(80, 22.132762909899988)
(96, 18.443714640327553)
(112, 15.807770214673472)
(128, 13.682439717504941)
(160, 10.981846176180943)
(192, 9.151476799986273)
(224, 7.874694345464264)
(256, 6.819452013960151)
(320, 5.463014408450611)
(384, 4.5380090994760875)
(448, 3.8910559472237027)
(512, 3.3682600715022786)
(640, 3.2634368557988527)
(768, 3.1132099898154357)
(896, 3.0877227011368222)
(1024, 2.2088726978773607)
(1280, 2.2962713591777213)
(1536, 1.819985722861657)
(1792, 1.9387351749381396)
(2048, 1.6268268641137789)
(2560, 1.5091002000131082)
(3072, 1.4292974298548107)
(3584, 1.373564416144417)
(4096, 1.331173514849102)
(5120, 1.2709834785081857)
(6144, 1.2308792848113586)
(7168, 1.2018766131732181)
(8192, 1.1799689449270265)
(10240, 1.1495567796920276)
(12288, 1.1280360135600562)
(14336, 1.114966914433802)
(16384, 1.104261500700414)
(20480, 1.0893195174998216)
(24576, 1.0764599514199007)
(28672, 1.0685280972064806)
(32768, 1.059348953471877)
(40960, 1.0512208664145624)
(49152, 1.0467292461602384)
(57344, 1.0402879896553108)
(65536, 1.0373360471270516)
(81920, 1.0329334618904913)
(98304, 1.0298398015265757)
(114688, 1.0272141508939268)
(131072, 1.025159687131917)
(163840, 1.0224255413214953)
(196608, 1.020630613454521)
(229376, 1.0211061349697177)
(262144, 1.0183487497800137)
(327680, 1.0174139978570318)
(393216, 1.0177281835721483)
(458752, 1.0166674805156883)
(524288, 1.0169733953976254)
(655360, 1.0164792743992346)
(786432, 1.0159576214316397)
(917504, 1.0189686552340584)
(1048576, 1.021260915603739)
};
% blake2sp
\addplot coordinates {
(64, 40.15854895574438)
(80, 32.28540860511629)
(96, 26.87249841806774)
(112, 23.140648497995535)
(128, 20.013859215679027)
(160, 16.10120393743009)
(192, 13.322730434708904)
(224, 11.476322610871755)
(256, 9.991253256846818)
(320, 7.924612486774671)
(384, 6.573308860288357)
(448, 5.5540357338208866)
(512, 4.771403610143878)
(640, 4.659056907993262)
(768, 4.038749378362821)
(896, 4.16686593825844)
(1024, 2.847769934030738)
(1280, 2.789947005748891)
(1536, 2.208958057916354)
(1792, 2.259252928161408)
(2048, 1.8892908312048184)
(2560, 1.6978567499339272)
(3072, 1.5698701432847812)
(3584, 1.478232494382148)
(4096, 1.4087154771705936)
(5120, 1.3107089037189212)
(6144, 1.2458733479330193)
(7168, 1.199972574236313)
(8192, 1.1651038552694755)
(10240, 1.1153542638702398)
(12288, 1.0829303005542)
(14336, 1.0584709682693951)
(16384, 1.0412212219468924)
(20480, 1.0174465589873738)
(24576, 1.0001432897858562)
(28672, 0.9887169821928306)
(32768, 0.9822754317133213)
(40960, 0.9699722101743756)
(49152, 0.9613047831247656)
(57344, 0.9549070811395712)
(65536, 0.9502962448742464)
(81920, 0.9444708054480934)
(98304, 0.9403464724733754)
(114688, 0.9373645976770237)
(131072, 0.9352568832368149)
(163840, 0.9320143218556695)
(196608, 0.9295512137629685)
(229376, 0.928001707535728)
(262144, 0.926457169683647)
(327680, 0.9248657193091026)
(393216, 0.9236790951734661)
(458752, 0.922680040786435)
(524288, 0.9222133976382022)
(655360, 0.9212799719308634)
(786432, 0.9210224257184921)
(917504, 0.9257740144609421)
(1048576, 0.9264588173936315)
};
% sha512
\addplot coordinates {
(64, 18.807588763929594)
(80, 15.055880488538238)
(96, 12.57560100978265)
(112, 17.965112609513262)
(128, 15.520415110627487)
(160, 12.466608818234963)
(192, 10.374034749410924)
(224, 8.913987582798113)
(256, 9.974198316354821)
(320, 8.005262549430489)
(384, 8.643464801651993)
(448, 7.439666719817537)
(512, 7.53541133106583)
(640, 7.21954245261229)
(768, 6.711731744365209)
(896, 6.608428491649153)
(1024, 6.30745856629269)
(1280, 6.059662579869737)
(1536, 5.910021396506304)
(1792, 5.771919101558227)
(2048, 5.700054094817165)
(2560, 5.5664409813913185)
(3072, 5.481442427408315)
(3584, 5.424451331903384)
(4096, 5.37259117599464)
(5120, 5.316686679313336)
(6144, 5.272001115893055)
(7168, 5.239077431418797)
(8192, 5.214135699223615)
(10240, 5.1807987999309)
(12288, 5.159673984906684)
(14336, 5.146421299704113)
(16384, 5.156329412613001)
(20480, 5.118131920599159)
(24576, 5.111780400017577)
(28672, 5.1068917200577655)
(32768, 5.101913859064512)
(40960, 5.115588446112568)
(49152, 5.098576596147654)
(57344, 5.090231423903652)
(65536, 5.086078384750424)
(81920, 5.084550062794866)
(98304, 5.088040046614853)
(114688, 5.083411511655933)
(131072, 5.082205007163851)
(163840, 5.087549997272495)
(196608, 5.086369216879108)
(229376, 5.082781610404937)
(262144, 5.083507522722811)
(327680, 5.081442137152084)
(393216, 5.224798618066921)
(458752, 5.270593404894116)
(524288, 5.2455733879013815)
(655360, 5.211609292633657)
(786432, 5.188971837312799)
(917504, 5.170556846457106)
(1048576, 5.161215564456634)
};
% sha256
\addplot coordinates {
(64, 22.39319136980306)
(80, 18.00628167013052)
(96, 14.989143757632228)
(112, 12.870459646112657)
(128, 14.770309151994706)
(160, 11.859920064452405)
(192, 12.563673408310647)
(224, 10.838076747051158)
(256, 11.188240455232643)
(320, 10.60041373270844)
(384, 9.988138485778947)
(448, 9.716250032048778)
(512, 9.408083361253391)
(640, 9.02658606507559)
(768, 8.789263455006406)
(896, 8.618090107135524)
(1024, 8.488869882067036)
(1280, 8.33031667948782)
(1536, 8.203874141208708)
(1792, 8.11399030570961)
(2048, 8.048206608328346)
(2560, 7.958167834029364)
(3072, 7.8989897294620315)
(3584, 7.8558550546969)
(4096, 7.830647251633207)
(5120, 7.780593038527965)
(6144, 7.751583463242047)
(7168, 7.726010491919893)
(8192, 7.709884963206814)
(10240, 7.683108023445339)
(12288, 7.668018376466106)
(14336, 7.665173571899678)
(16384, 7.662124459689593)
(20480, 7.657649434478918)
(24576, 7.664622007144224)
(28672, 7.668291931626329)
(32768, 7.681931266643737)
(40960, 7.64748847465036)
(49152, 7.659683656891216)
(57344, 7.655528399248478)
(65536, 7.634231892672145)
(81920, 7.636973246682833)
(98304, 7.625997455202883)
(114688, 7.63031378296365)
(131072, 7.625857374468353)
(163840, 7.625408080116657)
(196608, 7.618132024026258)
(229376, 7.624322176081686)
(262144, 7.8196342633939215)
(327680, 7.876313576268889)
(393216, 7.83378796940067)
(458752, 7.805312674192885)
(524288, 7.779588440209737)
(655360, 7.747810984425906)
(786432, 7.72389399329369)
(917504, 7.704285772158881)
(1048576, 7.691824030478032)
};
% kangarootwelve
\addplot coordinates {
(64, 9.121000714471853)
(80, 7.361066646016855)
(96, 6.1681076870526725)
(112, 5.303814578404992)
(128, 4.686930071084883)
(160, 3.7940168948718194)
(192, 5.21400924885899)
(224, 4.522517900055306)
(256, 3.9811367104288546)
(320, 3.2167987103762234)
(384, 3.6129076096320754)
(448, 3.0918228327937762)
(512, 3.4082920347251253)
(640, 2.778638273737875)
(768, 2.7946200537559123)
(896, 2.8058947093057207)
(1024, 2.8069941786771566)
(1280, 2.5233970601039615)
(1536, 2.572658272216899)
(1792, 2.4559996188464797)
(2048, 2.5057423317003127)
(2560, 2.4438092261716107)
(3072, 2.403370951617613)
(3584, 2.3742894737278117)
(4096, 2.352915836529019)
(5120, 2.322478517497599)
(6144, 2.3026346008339007)
(7168, 2.2882690700885067)
(8192, 2.395503077548445)
(10240, 2.3661407988718905)
(12288, 2.3378106457443075)
(14336, 2.3189275505552702)
(16384, 2.3330340272699397)
(20480, 2.3125855098334243)
(24576, 1.971953351896448)
(28672, 2.014050919354261)
(32768, 2.040401164807297)
(40960, 1.2316371300209783)
(49152, 1.4107958322383483)
(57344, 1.4145826594279836)
(65536, 1.5225132728284652)
(81920, 0.938208008112226)
(98304, 1.089182876106624)
(114688, 0.9481368912427958)
(131072, 1.0572553180607613)
(163840, 0.8892020751886123)
(196608, 0.9031859191730589)
(229376, 0.8066673375448304)
(262144, 0.8255207868065197)
(327680, 0.7802569430376484)
(393216, 0.7503076048553433)
(458752, 0.7279633948755418)
(524288, 0.7120508517636953)
(655360, 0.6878864628904291)
(786432, 0.6740100047522699)
(917504, 0.6679467780182821)
(1048576, 0.6624442975740585)
};
% horizontal rule
\addplot [black, dotted] coordinates {
(64, 0.49)
(1048576, 0.49)
};
\end{axis}
\end{tikzpicture}

\caption{Throughput for single-threaded BLAKE3 and other hash functions on an
    AWS c5.metal instance, measured in cycles per byte. This is an Intel
    Cascade Lake-SP processor with AVX-512 support. Lower is faster.}%
\label{fig:avx512}
\end{figure}

Figure~\ref{fig:avx512} shows the throughput of single-threaded BLAKE3 on
modern server hardware. This benchmark ran on an AWS c5.metal instance with a
pair of Intel Cascade Lake-SP 8275CL processors, which support AVX-512 vector
instructions.

The BLAKE3 curve passes through several different regimes in this figure. For a
single chunk of input, up to 1~KiB, BLAKE3 closely mirrors BLAKE2s (as it does
throughout all of Figure~\ref{fig:rpizero}). This is because their compression
functions are very similar, apart from the number of rounds. Between 1 and
4~KiB, the gap between BLAKE3 and BLAKE2s closes slightly. This is because
BLAKE3 begins dividing its input into multiple chunks with parent nodes, and
compressing the parent nodes adds overhead. At 4~KiB, BLAKE3 begins compressing
multiple chunks in parallel. Parallel chunk compression can use 128-bit SSE
vectors at 4~KiB, 256-bit AVX2 vectors at 8~KiB, and 512-bit AVX-512 vectors at
16~KiB. Beyond 16~KiB, throughput is relatively stable, until memory bandwidth
effects appear for very long inputs.

On this platform, the biggest performance difference between algorithms is how
much they can benefit from SIMD. Only BLAKE3 and KangarooTwelve are able to
take full advantage of AVX-512. The fixed tree structures of BLAKE2bp and
BLAKE2sp limit those algorithms to 256-bit vectors. Other hash functions are
much slower, because they cannot compress multiple blocks in parallel. BLAKE3
and KangarooTwelve are very close in peak throughput, but KangarooTwelve needs
more input to reach its peak, in part because of its larger 8~KiB chunk size.

\begin{figure}[h]
\centering
%%% Creator: Matplotlib, PGF backend
%%
%% To include the figure in your LaTeX document, write
%%   \input{<filename>.pgf}
%%
%% Make sure the required packages are loaded in your preamble
%%   \usepackage{pgf}
%%
%% Figures using additional raster images can only be included by \input if
%% they are in the same directory as the main LaTeX file. For loading figures
%% from other directories you can use the `import` package
%%   \usepackage{import}
%% and then include the figures with
%%   \import{<path to file>}{<filename>.pgf}
%%
%% Matplotlib used the following preamble
%%   \usepackage{fontspec}
%%   \setmainfont{DejaVuSerif.ttf}[Path=/usr/lib/python3.8/site-packages/matplotlib/mpl-data/fonts/ttf/]
%%   \setsansfont{DejaVuSans.ttf}[Path=/usr/lib/python3.8/site-packages/matplotlib/mpl-data/fonts/ttf/]
%%   \setmonofont{DejaVuSansMono.ttf}[Path=/usr/lib/python3.8/site-packages/matplotlib/mpl-data/fonts/ttf/]
%%
\begingroup%
\makeatletter%
\begin{pgfpicture}%
\pgfpathrectangle{\pgfpointorigin}{\pgfqpoint{5.889820in}{4.615557in}}%
\pgfusepath{use as bounding box, clip}%
\begin{pgfscope}%
\pgfsetbuttcap%
\pgfsetmiterjoin%
\definecolor{currentfill}{rgb}{1.000000,1.000000,1.000000}%
\pgfsetfillcolor{currentfill}%
\pgfsetlinewidth{0.000000pt}%
\definecolor{currentstroke}{rgb}{1.000000,1.000000,1.000000}%
\pgfsetstrokecolor{currentstroke}%
\pgfsetdash{}{0pt}%
\pgfpathmoveto{\pgfqpoint{0.000000in}{0.000000in}}%
\pgfpathlineto{\pgfqpoint{5.889820in}{0.000000in}}%
\pgfpathlineto{\pgfqpoint{5.889820in}{4.615557in}}%
\pgfpathlineto{\pgfqpoint{0.000000in}{4.615557in}}%
\pgfpathclose%
\pgfusepath{fill}%
\end{pgfscope}%
\begin{pgfscope}%
\pgfsetbuttcap%
\pgfsetmiterjoin%
\definecolor{currentfill}{rgb}{0.917647,0.917647,0.949020}%
\pgfsetfillcolor{currentfill}%
\pgfsetlinewidth{0.000000pt}%
\definecolor{currentstroke}{rgb}{0.000000,0.000000,0.000000}%
\pgfsetstrokecolor{currentstroke}%
\pgfsetstrokeopacity{0.000000}%
\pgfsetdash{}{0pt}%
\pgfpathmoveto{\pgfqpoint{0.829820in}{0.819557in}}%
\pgfpathlineto{\pgfqpoint{5.789820in}{0.819557in}}%
\pgfpathlineto{\pgfqpoint{5.789820in}{4.515557in}}%
\pgfpathlineto{\pgfqpoint{0.829820in}{4.515557in}}%
\pgfpathclose%
\pgfusepath{fill}%
\end{pgfscope}%
\begin{pgfscope}%
\pgfpathrectangle{\pgfqpoint{0.829820in}{0.819557in}}{\pgfqpoint{4.960000in}{3.696000in}}%
\pgfusepath{clip}%
\pgfsetroundcap%
\pgfsetroundjoin%
\pgfsetlinewidth{1.003750pt}%
\definecolor{currentstroke}{rgb}{1.000000,1.000000,1.000000}%
\pgfsetstrokecolor{currentstroke}%
\pgfsetdash{}{0pt}%
\pgfpathmoveto{\pgfqpoint{1.055275in}{0.819557in}}%
\pgfpathlineto{\pgfqpoint{1.055275in}{4.515557in}}%
\pgfusepath{stroke}%
\end{pgfscope}%
\begin{pgfscope}%
\definecolor{textcolor}{rgb}{0.150000,0.150000,0.150000}%
\pgfsetstrokecolor{textcolor}%
\pgfsetfillcolor{textcolor}%
\pgftext[x=1.013126in,y=0.687613in,left,base,rotate=270.000000]{\color{textcolor}\sffamily\fontsize{11.000000}{13.200000}\selectfont 64}%
\end{pgfscope}%
\begin{pgfscope}%
\pgfpathrectangle{\pgfqpoint{0.829820in}{0.819557in}}{\pgfqpoint{4.960000in}{3.696000in}}%
\pgfusepath{clip}%
\pgfsetroundcap%
\pgfsetroundjoin%
\pgfsetlinewidth{1.003750pt}%
\definecolor{currentstroke}{rgb}{1.000000,1.000000,1.000000}%
\pgfsetstrokecolor{currentstroke}%
\pgfsetdash{}{0pt}%
\pgfpathmoveto{\pgfqpoint{1.377353in}{0.819557in}}%
\pgfpathlineto{\pgfqpoint{1.377353in}{4.515557in}}%
\pgfusepath{stroke}%
\end{pgfscope}%
\begin{pgfscope}%
\definecolor{textcolor}{rgb}{0.150000,0.150000,0.150000}%
\pgfsetstrokecolor{textcolor}%
\pgfsetfillcolor{textcolor}%
\pgftext[x=1.335204in,y=0.687613in,left,base,rotate=270.000000]{\color{textcolor}\sffamily\fontsize{11.000000}{13.200000}\selectfont 128}%
\end{pgfscope}%
\begin{pgfscope}%
\pgfpathrectangle{\pgfqpoint{0.829820in}{0.819557in}}{\pgfqpoint{4.960000in}{3.696000in}}%
\pgfusepath{clip}%
\pgfsetroundcap%
\pgfsetroundjoin%
\pgfsetlinewidth{1.003750pt}%
\definecolor{currentstroke}{rgb}{1.000000,1.000000,1.000000}%
\pgfsetstrokecolor{currentstroke}%
\pgfsetdash{}{0pt}%
\pgfpathmoveto{\pgfqpoint{1.699431in}{0.819557in}}%
\pgfpathlineto{\pgfqpoint{1.699431in}{4.515557in}}%
\pgfusepath{stroke}%
\end{pgfscope}%
\begin{pgfscope}%
\definecolor{textcolor}{rgb}{0.150000,0.150000,0.150000}%
\pgfsetstrokecolor{textcolor}%
\pgfsetfillcolor{textcolor}%
\pgftext[x=1.657282in,y=0.687613in,left,base,rotate=270.000000]{\color{textcolor}\sffamily\fontsize{11.000000}{13.200000}\selectfont 256}%
\end{pgfscope}%
\begin{pgfscope}%
\pgfpathrectangle{\pgfqpoint{0.829820in}{0.819557in}}{\pgfqpoint{4.960000in}{3.696000in}}%
\pgfusepath{clip}%
\pgfsetroundcap%
\pgfsetroundjoin%
\pgfsetlinewidth{1.003750pt}%
\definecolor{currentstroke}{rgb}{1.000000,1.000000,1.000000}%
\pgfsetstrokecolor{currentstroke}%
\pgfsetdash{}{0pt}%
\pgfpathmoveto{\pgfqpoint{2.021508in}{0.819557in}}%
\pgfpathlineto{\pgfqpoint{2.021508in}{4.515557in}}%
\pgfusepath{stroke}%
\end{pgfscope}%
\begin{pgfscope}%
\definecolor{textcolor}{rgb}{0.150000,0.150000,0.150000}%
\pgfsetstrokecolor{textcolor}%
\pgfsetfillcolor{textcolor}%
\pgftext[x=1.979360in,y=0.687613in,left,base,rotate=270.000000]{\color{textcolor}\sffamily\fontsize{11.000000}{13.200000}\selectfont 512}%
\end{pgfscope}%
\begin{pgfscope}%
\pgfpathrectangle{\pgfqpoint{0.829820in}{0.819557in}}{\pgfqpoint{4.960000in}{3.696000in}}%
\pgfusepath{clip}%
\pgfsetroundcap%
\pgfsetroundjoin%
\pgfsetlinewidth{1.003750pt}%
\definecolor{currentstroke}{rgb}{1.000000,1.000000,1.000000}%
\pgfsetstrokecolor{currentstroke}%
\pgfsetdash{}{0pt}%
\pgfpathmoveto{\pgfqpoint{2.343586in}{0.819557in}}%
\pgfpathlineto{\pgfqpoint{2.343586in}{4.515557in}}%
\pgfusepath{stroke}%
\end{pgfscope}%
\begin{pgfscope}%
\definecolor{textcolor}{rgb}{0.150000,0.150000,0.150000}%
\pgfsetstrokecolor{textcolor}%
\pgfsetfillcolor{textcolor}%
\pgftext[x=2.301438in,y=0.687613in,left,base,rotate=270.000000]{\color{textcolor}\sffamily\fontsize{11.000000}{13.200000}\selectfont 1 KiB}%
\end{pgfscope}%
\begin{pgfscope}%
\pgfpathrectangle{\pgfqpoint{0.829820in}{0.819557in}}{\pgfqpoint{4.960000in}{3.696000in}}%
\pgfusepath{clip}%
\pgfsetroundcap%
\pgfsetroundjoin%
\pgfsetlinewidth{1.003750pt}%
\definecolor{currentstroke}{rgb}{1.000000,1.000000,1.000000}%
\pgfsetstrokecolor{currentstroke}%
\pgfsetdash{}{0pt}%
\pgfpathmoveto{\pgfqpoint{2.665664in}{0.819557in}}%
\pgfpathlineto{\pgfqpoint{2.665664in}{4.515557in}}%
\pgfusepath{stroke}%
\end{pgfscope}%
\begin{pgfscope}%
\definecolor{textcolor}{rgb}{0.150000,0.150000,0.150000}%
\pgfsetstrokecolor{textcolor}%
\pgfsetfillcolor{textcolor}%
\pgftext[x=2.623516in,y=0.687613in,left,base,rotate=270.000000]{\color{textcolor}\sffamily\fontsize{11.000000}{13.200000}\selectfont 2 KiB}%
\end{pgfscope}%
\begin{pgfscope}%
\pgfpathrectangle{\pgfqpoint{0.829820in}{0.819557in}}{\pgfqpoint{4.960000in}{3.696000in}}%
\pgfusepath{clip}%
\pgfsetroundcap%
\pgfsetroundjoin%
\pgfsetlinewidth{1.003750pt}%
\definecolor{currentstroke}{rgb}{1.000000,1.000000,1.000000}%
\pgfsetstrokecolor{currentstroke}%
\pgfsetdash{}{0pt}%
\pgfpathmoveto{\pgfqpoint{2.987742in}{0.819557in}}%
\pgfpathlineto{\pgfqpoint{2.987742in}{4.515557in}}%
\pgfusepath{stroke}%
\end{pgfscope}%
\begin{pgfscope}%
\definecolor{textcolor}{rgb}{0.150000,0.150000,0.150000}%
\pgfsetstrokecolor{textcolor}%
\pgfsetfillcolor{textcolor}%
\pgftext[x=2.945594in,y=0.687613in,left,base,rotate=270.000000]{\color{textcolor}\sffamily\fontsize{11.000000}{13.200000}\selectfont 4 KiB}%
\end{pgfscope}%
\begin{pgfscope}%
\pgfpathrectangle{\pgfqpoint{0.829820in}{0.819557in}}{\pgfqpoint{4.960000in}{3.696000in}}%
\pgfusepath{clip}%
\pgfsetroundcap%
\pgfsetroundjoin%
\pgfsetlinewidth{1.003750pt}%
\definecolor{currentstroke}{rgb}{1.000000,1.000000,1.000000}%
\pgfsetstrokecolor{currentstroke}%
\pgfsetdash{}{0pt}%
\pgfpathmoveto{\pgfqpoint{3.309820in}{0.819557in}}%
\pgfpathlineto{\pgfqpoint{3.309820in}{4.515557in}}%
\pgfusepath{stroke}%
\end{pgfscope}%
\begin{pgfscope}%
\definecolor{textcolor}{rgb}{0.150000,0.150000,0.150000}%
\pgfsetstrokecolor{textcolor}%
\pgfsetfillcolor{textcolor}%
\pgftext[x=3.267672in,y=0.687613in,left,base,rotate=270.000000]{\color{textcolor}\sffamily\fontsize{11.000000}{13.200000}\selectfont 8 KiB}%
\end{pgfscope}%
\begin{pgfscope}%
\pgfpathrectangle{\pgfqpoint{0.829820in}{0.819557in}}{\pgfqpoint{4.960000in}{3.696000in}}%
\pgfusepath{clip}%
\pgfsetroundcap%
\pgfsetroundjoin%
\pgfsetlinewidth{1.003750pt}%
\definecolor{currentstroke}{rgb}{1.000000,1.000000,1.000000}%
\pgfsetstrokecolor{currentstroke}%
\pgfsetdash{}{0pt}%
\pgfpathmoveto{\pgfqpoint{3.631898in}{0.819557in}}%
\pgfpathlineto{\pgfqpoint{3.631898in}{4.515557in}}%
\pgfusepath{stroke}%
\end{pgfscope}%
\begin{pgfscope}%
\definecolor{textcolor}{rgb}{0.150000,0.150000,0.150000}%
\pgfsetstrokecolor{textcolor}%
\pgfsetfillcolor{textcolor}%
\pgftext[x=3.589750in,y=0.687613in,left,base,rotate=270.000000]{\color{textcolor}\sffamily\fontsize{11.000000}{13.200000}\selectfont 16 KiB}%
\end{pgfscope}%
\begin{pgfscope}%
\pgfpathrectangle{\pgfqpoint{0.829820in}{0.819557in}}{\pgfqpoint{4.960000in}{3.696000in}}%
\pgfusepath{clip}%
\pgfsetroundcap%
\pgfsetroundjoin%
\pgfsetlinewidth{1.003750pt}%
\definecolor{currentstroke}{rgb}{1.000000,1.000000,1.000000}%
\pgfsetstrokecolor{currentstroke}%
\pgfsetdash{}{0pt}%
\pgfpathmoveto{\pgfqpoint{3.953976in}{0.819557in}}%
\pgfpathlineto{\pgfqpoint{3.953976in}{4.515557in}}%
\pgfusepath{stroke}%
\end{pgfscope}%
\begin{pgfscope}%
\definecolor{textcolor}{rgb}{0.150000,0.150000,0.150000}%
\pgfsetstrokecolor{textcolor}%
\pgfsetfillcolor{textcolor}%
\pgftext[x=3.911828in,y=0.687613in,left,base,rotate=270.000000]{\color{textcolor}\sffamily\fontsize{11.000000}{13.200000}\selectfont 32 KiB}%
\end{pgfscope}%
\begin{pgfscope}%
\pgfpathrectangle{\pgfqpoint{0.829820in}{0.819557in}}{\pgfqpoint{4.960000in}{3.696000in}}%
\pgfusepath{clip}%
\pgfsetroundcap%
\pgfsetroundjoin%
\pgfsetlinewidth{1.003750pt}%
\definecolor{currentstroke}{rgb}{1.000000,1.000000,1.000000}%
\pgfsetstrokecolor{currentstroke}%
\pgfsetdash{}{0pt}%
\pgfpathmoveto{\pgfqpoint{4.276054in}{0.819557in}}%
\pgfpathlineto{\pgfqpoint{4.276054in}{4.515557in}}%
\pgfusepath{stroke}%
\end{pgfscope}%
\begin{pgfscope}%
\definecolor{textcolor}{rgb}{0.150000,0.150000,0.150000}%
\pgfsetstrokecolor{textcolor}%
\pgfsetfillcolor{textcolor}%
\pgftext[x=4.233906in,y=0.687613in,left,base,rotate=270.000000]{\color{textcolor}\sffamily\fontsize{11.000000}{13.200000}\selectfont 64 KiB}%
\end{pgfscope}%
\begin{pgfscope}%
\pgfpathrectangle{\pgfqpoint{0.829820in}{0.819557in}}{\pgfqpoint{4.960000in}{3.696000in}}%
\pgfusepath{clip}%
\pgfsetroundcap%
\pgfsetroundjoin%
\pgfsetlinewidth{1.003750pt}%
\definecolor{currentstroke}{rgb}{1.000000,1.000000,1.000000}%
\pgfsetstrokecolor{currentstroke}%
\pgfsetdash{}{0pt}%
\pgfpathmoveto{\pgfqpoint{4.598132in}{0.819557in}}%
\pgfpathlineto{\pgfqpoint{4.598132in}{4.515557in}}%
\pgfusepath{stroke}%
\end{pgfscope}%
\begin{pgfscope}%
\definecolor{textcolor}{rgb}{0.150000,0.150000,0.150000}%
\pgfsetstrokecolor{textcolor}%
\pgfsetfillcolor{textcolor}%
\pgftext[x=4.555984in,y=0.687613in,left,base,rotate=270.000000]{\color{textcolor}\sffamily\fontsize{11.000000}{13.200000}\selectfont 128 KiB}%
\end{pgfscope}%
\begin{pgfscope}%
\pgfpathrectangle{\pgfqpoint{0.829820in}{0.819557in}}{\pgfqpoint{4.960000in}{3.696000in}}%
\pgfusepath{clip}%
\pgfsetroundcap%
\pgfsetroundjoin%
\pgfsetlinewidth{1.003750pt}%
\definecolor{currentstroke}{rgb}{1.000000,1.000000,1.000000}%
\pgfsetstrokecolor{currentstroke}%
\pgfsetdash{}{0pt}%
\pgfpathmoveto{\pgfqpoint{4.920210in}{0.819557in}}%
\pgfpathlineto{\pgfqpoint{4.920210in}{4.515557in}}%
\pgfusepath{stroke}%
\end{pgfscope}%
\begin{pgfscope}%
\definecolor{textcolor}{rgb}{0.150000,0.150000,0.150000}%
\pgfsetstrokecolor{textcolor}%
\pgfsetfillcolor{textcolor}%
\pgftext[x=4.878061in,y=0.687613in,left,base,rotate=270.000000]{\color{textcolor}\sffamily\fontsize{11.000000}{13.200000}\selectfont 256 KiB}%
\end{pgfscope}%
\begin{pgfscope}%
\pgfpathrectangle{\pgfqpoint{0.829820in}{0.819557in}}{\pgfqpoint{4.960000in}{3.696000in}}%
\pgfusepath{clip}%
\pgfsetroundcap%
\pgfsetroundjoin%
\pgfsetlinewidth{1.003750pt}%
\definecolor{currentstroke}{rgb}{1.000000,1.000000,1.000000}%
\pgfsetstrokecolor{currentstroke}%
\pgfsetdash{}{0pt}%
\pgfpathmoveto{\pgfqpoint{5.242288in}{0.819557in}}%
\pgfpathlineto{\pgfqpoint{5.242288in}{4.515557in}}%
\pgfusepath{stroke}%
\end{pgfscope}%
\begin{pgfscope}%
\definecolor{textcolor}{rgb}{0.150000,0.150000,0.150000}%
\pgfsetstrokecolor{textcolor}%
\pgfsetfillcolor{textcolor}%
\pgftext[x=5.200139in,y=0.687613in,left,base,rotate=270.000000]{\color{textcolor}\sffamily\fontsize{11.000000}{13.200000}\selectfont 512 KiB}%
\end{pgfscope}%
\begin{pgfscope}%
\pgfpathrectangle{\pgfqpoint{0.829820in}{0.819557in}}{\pgfqpoint{4.960000in}{3.696000in}}%
\pgfusepath{clip}%
\pgfsetroundcap%
\pgfsetroundjoin%
\pgfsetlinewidth{1.003750pt}%
\definecolor{currentstroke}{rgb}{1.000000,1.000000,1.000000}%
\pgfsetstrokecolor{currentstroke}%
\pgfsetdash{}{0pt}%
\pgfpathmoveto{\pgfqpoint{5.564366in}{0.819557in}}%
\pgfpathlineto{\pgfqpoint{5.564366in}{4.515557in}}%
\pgfusepath{stroke}%
\end{pgfscope}%
\begin{pgfscope}%
\definecolor{textcolor}{rgb}{0.150000,0.150000,0.150000}%
\pgfsetstrokecolor{textcolor}%
\pgfsetfillcolor{textcolor}%
\pgftext[x=5.522217in,y=0.687613in,left,base,rotate=270.000000]{\color{textcolor}\sffamily\fontsize{11.000000}{13.200000}\selectfont 1 MiB}%
\end{pgfscope}%
\begin{pgfscope}%
\pgfpathrectangle{\pgfqpoint{0.829820in}{0.819557in}}{\pgfqpoint{4.960000in}{3.696000in}}%
\pgfusepath{clip}%
\pgfsetroundcap%
\pgfsetroundjoin%
\pgfsetlinewidth{1.003750pt}%
\definecolor{currentstroke}{rgb}{1.000000,1.000000,1.000000}%
\pgfsetstrokecolor{currentstroke}%
\pgfsetdash{}{0pt}%
\pgfpathmoveto{\pgfqpoint{0.829820in}{0.819557in}}%
\pgfpathlineto{\pgfqpoint{5.789820in}{0.819557in}}%
\pgfusepath{stroke}%
\end{pgfscope}%
\begin{pgfscope}%
\definecolor{textcolor}{rgb}{0.150000,0.150000,0.150000}%
\pgfsetstrokecolor{textcolor}%
\pgfsetfillcolor{textcolor}%
\pgftext[x=0.600674in,y=0.761519in,left,base]{\color{textcolor}\sffamily\fontsize{11.000000}{13.200000}\selectfont 0}%
\end{pgfscope}%
\begin{pgfscope}%
\pgfpathrectangle{\pgfqpoint{0.829820in}{0.819557in}}{\pgfqpoint{4.960000in}{3.696000in}}%
\pgfusepath{clip}%
\pgfsetroundcap%
\pgfsetroundjoin%
\pgfsetlinewidth{1.003750pt}%
\definecolor{currentstroke}{rgb}{1.000000,1.000000,1.000000}%
\pgfsetstrokecolor{currentstroke}%
\pgfsetdash{}{0pt}%
\pgfpathmoveto{\pgfqpoint{0.829820in}{1.574567in}}%
\pgfpathlineto{\pgfqpoint{5.789820in}{1.574567in}}%
\pgfusepath{stroke}%
\end{pgfscope}%
\begin{pgfscope}%
\definecolor{textcolor}{rgb}{0.150000,0.150000,0.150000}%
\pgfsetstrokecolor{textcolor}%
\pgfsetfillcolor{textcolor}%
\pgftext[x=0.503472in,y=1.516529in,left,base]{\color{textcolor}\sffamily\fontsize{11.000000}{13.200000}\selectfont 10}%
\end{pgfscope}%
\begin{pgfscope}%
\pgfpathrectangle{\pgfqpoint{0.829820in}{0.819557in}}{\pgfqpoint{4.960000in}{3.696000in}}%
\pgfusepath{clip}%
\pgfsetroundcap%
\pgfsetroundjoin%
\pgfsetlinewidth{1.003750pt}%
\definecolor{currentstroke}{rgb}{1.000000,1.000000,1.000000}%
\pgfsetstrokecolor{currentstroke}%
\pgfsetdash{}{0pt}%
\pgfpathmoveto{\pgfqpoint{0.829820in}{2.329577in}}%
\pgfpathlineto{\pgfqpoint{5.789820in}{2.329577in}}%
\pgfusepath{stroke}%
\end{pgfscope}%
\begin{pgfscope}%
\definecolor{textcolor}{rgb}{0.150000,0.150000,0.150000}%
\pgfsetstrokecolor{textcolor}%
\pgfsetfillcolor{textcolor}%
\pgftext[x=0.503472in,y=2.271539in,left,base]{\color{textcolor}\sffamily\fontsize{11.000000}{13.200000}\selectfont 20}%
\end{pgfscope}%
\begin{pgfscope}%
\pgfpathrectangle{\pgfqpoint{0.829820in}{0.819557in}}{\pgfqpoint{4.960000in}{3.696000in}}%
\pgfusepath{clip}%
\pgfsetroundcap%
\pgfsetroundjoin%
\pgfsetlinewidth{1.003750pt}%
\definecolor{currentstroke}{rgb}{1.000000,1.000000,1.000000}%
\pgfsetstrokecolor{currentstroke}%
\pgfsetdash{}{0pt}%
\pgfpathmoveto{\pgfqpoint{0.829820in}{3.084587in}}%
\pgfpathlineto{\pgfqpoint{5.789820in}{3.084587in}}%
\pgfusepath{stroke}%
\end{pgfscope}%
\begin{pgfscope}%
\definecolor{textcolor}{rgb}{0.150000,0.150000,0.150000}%
\pgfsetstrokecolor{textcolor}%
\pgfsetfillcolor{textcolor}%
\pgftext[x=0.503472in,y=3.026549in,left,base]{\color{textcolor}\sffamily\fontsize{11.000000}{13.200000}\selectfont 30}%
\end{pgfscope}%
\begin{pgfscope}%
\pgfpathrectangle{\pgfqpoint{0.829820in}{0.819557in}}{\pgfqpoint{4.960000in}{3.696000in}}%
\pgfusepath{clip}%
\pgfsetroundcap%
\pgfsetroundjoin%
\pgfsetlinewidth{1.003750pt}%
\definecolor{currentstroke}{rgb}{1.000000,1.000000,1.000000}%
\pgfsetstrokecolor{currentstroke}%
\pgfsetdash{}{0pt}%
\pgfpathmoveto{\pgfqpoint{0.829820in}{3.839597in}}%
\pgfpathlineto{\pgfqpoint{5.789820in}{3.839597in}}%
\pgfusepath{stroke}%
\end{pgfscope}%
\begin{pgfscope}%
\definecolor{textcolor}{rgb}{0.150000,0.150000,0.150000}%
\pgfsetstrokecolor{textcolor}%
\pgfsetfillcolor{textcolor}%
\pgftext[x=0.503472in,y=3.781559in,left,base]{\color{textcolor}\sffamily\fontsize{11.000000}{13.200000}\selectfont 40}%
\end{pgfscope}%
\begin{pgfscope}%
\definecolor{textcolor}{rgb}{0.150000,0.150000,0.150000}%
\pgfsetstrokecolor{textcolor}%
\pgfsetfillcolor{textcolor}%
\pgftext[x=0.226628in,y=1.891434in,left,base,rotate=90.000000]{\color{textcolor}\sffamily\fontsize{12.000000}{14.400000}\selectfont Throughput (MB/s)}%
\end{pgfscope}%
\begin{pgfscope}%
\definecolor{textcolor}{rgb}{0.150000,0.150000,0.150000}%
\pgfsetstrokecolor{textcolor}%
\pgfsetfillcolor{textcolor}%
\pgftext[x=0.413249in,y=2.667557in,left,base,rotate=90.000000]{\color{textcolor}\sffamily\fontsize{12.000000}{14.400000}\selectfont }%
\end{pgfscope}%
\begin{pgfscope}%
\pgfpathrectangle{\pgfqpoint{0.829820in}{0.819557in}}{\pgfqpoint{4.960000in}{3.696000in}}%
\pgfusepath{clip}%
\pgfsetroundcap%
\pgfsetroundjoin%
\pgfsetlinewidth{1.204500pt}%
\definecolor{currentstroke}{rgb}{0.298039,0.447059,0.690196}%
\pgfsetstrokecolor{currentstroke}%
\pgfsetdash{}{0pt}%
\pgfpathmoveto{\pgfqpoint{1.055275in}{3.428925in}}%
\pgfpathlineto{\pgfqpoint{1.377353in}{3.776121in}}%
\pgfpathlineto{\pgfqpoint{1.699431in}{3.982122in}}%
\pgfpathlineto{\pgfqpoint{2.021508in}{4.091036in}}%
\pgfpathlineto{\pgfqpoint{2.343586in}{4.150203in}}%
\pgfpathlineto{\pgfqpoint{2.665664in}{4.179557in}}%
\pgfpathlineto{\pgfqpoint{2.987742in}{4.133355in}}%
\pgfpathlineto{\pgfqpoint{3.309820in}{4.103345in}}%
\pgfpathlineto{\pgfqpoint{3.631898in}{4.000178in}}%
\pgfpathlineto{\pgfqpoint{3.953976in}{3.887807in}}%
\pgfpathlineto{\pgfqpoint{4.276054in}{3.830207in}}%
\pgfpathlineto{\pgfqpoint{4.598132in}{3.641502in}}%
\pgfpathlineto{\pgfqpoint{4.920210in}{3.550683in}}%
\pgfpathlineto{\pgfqpoint{5.242288in}{3.548626in}}%
\pgfpathlineto{\pgfqpoint{5.564366in}{3.549687in}}%
\pgfusepath{stroke}%
\end{pgfscope}%
\begin{pgfscope}%
\pgfpathrectangle{\pgfqpoint{0.829820in}{0.819557in}}{\pgfqpoint{4.960000in}{3.696000in}}%
\pgfusepath{clip}%
\pgfsetbuttcap%
\pgfsetroundjoin%
\pgfsetlinewidth{1.204500pt}%
\definecolor{currentstroke}{rgb}{0.866667,0.517647,0.321569}%
\pgfsetstrokecolor{currentstroke}%
\pgfsetdash{{4.800000pt}{1.800000pt}}{0.000000pt}%
\pgfpathmoveto{\pgfqpoint{1.055275in}{2.819837in}}%
\pgfpathlineto{\pgfqpoint{1.377353in}{3.005416in}}%
\pgfpathlineto{\pgfqpoint{1.699431in}{3.115698in}}%
\pgfpathlineto{\pgfqpoint{2.021508in}{3.176818in}}%
\pgfpathlineto{\pgfqpoint{2.343586in}{3.203617in}}%
\pgfpathlineto{\pgfqpoint{2.665664in}{3.223501in}}%
\pgfpathlineto{\pgfqpoint{2.987742in}{3.232716in}}%
\pgfpathlineto{\pgfqpoint{3.309820in}{3.227375in}}%
\pgfpathlineto{\pgfqpoint{3.631898in}{3.204427in}}%
\pgfpathlineto{\pgfqpoint{3.953976in}{3.134911in}}%
\pgfpathlineto{\pgfqpoint{4.276054in}{3.096956in}}%
\pgfpathlineto{\pgfqpoint{4.598132in}{2.991162in}}%
\pgfpathlineto{\pgfqpoint{4.920210in}{2.936594in}}%
\pgfpathlineto{\pgfqpoint{5.242288in}{2.937191in}}%
\pgfpathlineto{\pgfqpoint{5.564366in}{2.938183in}}%
\pgfusepath{stroke}%
\end{pgfscope}%
\begin{pgfscope}%
\pgfpathrectangle{\pgfqpoint{0.829820in}{0.819557in}}{\pgfqpoint{4.960000in}{3.696000in}}%
\pgfusepath{clip}%
\pgfsetbuttcap%
\pgfsetroundjoin%
\pgfsetlinewidth{1.204500pt}%
\definecolor{currentstroke}{rgb}{0.333333,0.658824,0.407843}%
\pgfsetstrokecolor{currentstroke}%
\pgfsetdash{{1.200000pt}{1.200000pt}}{0.000000pt}%
\pgfpathmoveto{\pgfqpoint{1.055275in}{1.266912in}}%
\pgfpathlineto{\pgfqpoint{1.377353in}{1.736768in}}%
\pgfpathlineto{\pgfqpoint{1.699431in}{1.773448in}}%
\pgfpathlineto{\pgfqpoint{2.021508in}{1.796146in}}%
\pgfpathlineto{\pgfqpoint{2.343586in}{1.804082in}}%
\pgfpathlineto{\pgfqpoint{2.665664in}{1.810832in}}%
\pgfpathlineto{\pgfqpoint{2.987742in}{1.811936in}}%
\pgfpathlineto{\pgfqpoint{3.309820in}{1.810854in}}%
\pgfpathlineto{\pgfqpoint{3.631898in}{1.802851in}}%
\pgfpathlineto{\pgfqpoint{3.953976in}{1.792847in}}%
\pgfpathlineto{\pgfqpoint{4.276054in}{1.784067in}}%
\pgfpathlineto{\pgfqpoint{4.598132in}{1.762915in}}%
\pgfpathlineto{\pgfqpoint{4.920210in}{1.756572in}}%
\pgfpathlineto{\pgfqpoint{5.242288in}{1.756480in}}%
\pgfpathlineto{\pgfqpoint{5.564366in}{1.756856in}}%
\pgfusepath{stroke}%
\end{pgfscope}%
\begin{pgfscope}%
\pgfpathrectangle{\pgfqpoint{0.829820in}{0.819557in}}{\pgfqpoint{4.960000in}{3.696000in}}%
\pgfusepath{clip}%
\pgfsetbuttcap%
\pgfsetroundjoin%
\pgfsetlinewidth{1.204500pt}%
\definecolor{currentstroke}{rgb}{0.768627,0.305882,0.321569}%
\pgfsetstrokecolor{currentstroke}%
\pgfsetdash{{3.600000pt}{1.200000pt}{1.800000pt}{1.200000pt}}{0.000000pt}%
\pgfpathmoveto{\pgfqpoint{1.055275in}{0.991582in}}%
\pgfpathlineto{\pgfqpoint{1.377353in}{1.163676in}}%
\pgfpathlineto{\pgfqpoint{1.699431in}{1.511218in}}%
\pgfpathlineto{\pgfqpoint{2.021508in}{2.215669in}}%
\pgfpathlineto{\pgfqpoint{2.343586in}{2.590804in}}%
\pgfpathlineto{\pgfqpoint{2.665664in}{2.863207in}}%
\pgfpathlineto{\pgfqpoint{2.987742in}{3.033050in}}%
\pgfpathlineto{\pgfqpoint{3.309820in}{3.122826in}}%
\pgfpathlineto{\pgfqpoint{3.631898in}{3.130394in}}%
\pgfpathlineto{\pgfqpoint{3.953976in}{3.068117in}}%
\pgfpathlineto{\pgfqpoint{4.276054in}{2.973363in}}%
\pgfpathlineto{\pgfqpoint{4.598132in}{2.897437in}}%
\pgfpathlineto{\pgfqpoint{4.920210in}{2.835984in}}%
\pgfpathlineto{\pgfqpoint{5.242288in}{2.840747in}}%
\pgfpathlineto{\pgfqpoint{5.564366in}{2.845754in}}%
\pgfusepath{stroke}%
\end{pgfscope}%
\begin{pgfscope}%
\pgfpathrectangle{\pgfqpoint{0.829820in}{0.819557in}}{\pgfqpoint{4.960000in}{3.696000in}}%
\pgfusepath{clip}%
\pgfsetbuttcap%
\pgfsetroundjoin%
\pgfsetlinewidth{1.204500pt}%
\definecolor{currentstroke}{rgb}{0.505882,0.447059,0.701961}%
\pgfsetstrokecolor{currentstroke}%
\pgfsetdash{{6.000000pt}{1.200000pt}{1.200000pt}{1.200000pt}}{0.000000pt}%
\pgfpathmoveto{\pgfqpoint{1.055275in}{0.896867in}}%
\pgfpathlineto{\pgfqpoint{1.377353in}{0.974047in}}%
\pgfpathlineto{\pgfqpoint{1.699431in}{1.129442in}}%
\pgfpathlineto{\pgfqpoint{2.021508in}{1.440663in}}%
\pgfpathlineto{\pgfqpoint{2.343586in}{1.584583in}}%
\pgfpathlineto{\pgfqpoint{2.665664in}{1.684570in}}%
\pgfpathlineto{\pgfqpoint{2.987742in}{1.743279in}}%
\pgfpathlineto{\pgfqpoint{3.309820in}{1.776925in}}%
\pgfpathlineto{\pgfqpoint{3.631898in}{1.780143in}}%
\pgfpathlineto{\pgfqpoint{3.953976in}{1.780557in}}%
\pgfpathlineto{\pgfqpoint{4.276054in}{1.771647in}}%
\pgfpathlineto{\pgfqpoint{4.598132in}{1.750783in}}%
\pgfpathlineto{\pgfqpoint{4.920210in}{1.744819in}}%
\pgfpathlineto{\pgfqpoint{5.242288in}{1.746753in}}%
\pgfpathlineto{\pgfqpoint{5.564366in}{1.748245in}}%
\pgfusepath{stroke}%
\end{pgfscope}%
\begin{pgfscope}%
\pgfpathrectangle{\pgfqpoint{0.829820in}{0.819557in}}{\pgfqpoint{4.960000in}{3.696000in}}%
\pgfusepath{clip}%
\pgfsetbuttcap%
\pgfsetroundjoin%
\pgfsetlinewidth{1.204500pt}%
\definecolor{currentstroke}{rgb}{0.576471,0.470588,0.376471}%
\pgfsetstrokecolor{currentstroke}%
\pgfsetdash{{6.000000pt}{1.200000pt}{2.400000pt}{1.200000pt}{2.400000pt}{1.200000pt}}{0.000000pt}%
\pgfpathmoveto{\pgfqpoint{1.055275in}{1.390509in}}%
\pgfpathlineto{\pgfqpoint{1.377353in}{1.751932in}}%
\pgfpathlineto{\pgfqpoint{1.699431in}{2.090818in}}%
\pgfpathlineto{\pgfqpoint{2.021508in}{2.485518in}}%
\pgfpathlineto{\pgfqpoint{2.343586in}{2.762500in}}%
\pgfpathlineto{\pgfqpoint{2.665664in}{2.944864in}}%
\pgfpathlineto{\pgfqpoint{2.987742in}{3.040753in}}%
\pgfpathlineto{\pgfqpoint{3.309820in}{3.084834in}}%
\pgfpathlineto{\pgfqpoint{3.631898in}{3.077800in}}%
\pgfpathlineto{\pgfqpoint{3.953976in}{3.047278in}}%
\pgfpathlineto{\pgfqpoint{4.276054in}{3.024958in}}%
\pgfpathlineto{\pgfqpoint{4.598132in}{2.934120in}}%
\pgfpathlineto{\pgfqpoint{4.920210in}{2.882389in}}%
\pgfpathlineto{\pgfqpoint{5.242288in}{2.887568in}}%
\pgfpathlineto{\pgfqpoint{5.564366in}{2.888758in}}%
\pgfusepath{stroke}%
\end{pgfscope}%
\begin{pgfscope}%
\pgfpathrectangle{\pgfqpoint{0.829820in}{0.819557in}}{\pgfqpoint{4.960000in}{3.696000in}}%
\pgfusepath{clip}%
\pgfsetbuttcap%
\pgfsetroundjoin%
\pgfsetlinewidth{1.204500pt}%
\definecolor{currentstroke}{rgb}{0.854902,0.545098,0.764706}%
\pgfsetstrokecolor{currentstroke}%
\pgfsetdash{{2.400000pt}{2.400000pt}{3.600000pt}{1.800000pt}}{0.000000pt}%
\pgfpathmoveto{\pgfqpoint{1.055275in}{1.167652in}}%
\pgfpathlineto{\pgfqpoint{1.377353in}{1.247645in}}%
\pgfpathlineto{\pgfqpoint{1.699431in}{1.422749in}}%
\pgfpathlineto{\pgfqpoint{2.021508in}{1.590710in}}%
\pgfpathlineto{\pgfqpoint{2.343586in}{1.709075in}}%
\pgfpathlineto{\pgfqpoint{2.665664in}{1.785668in}}%
\pgfpathlineto{\pgfqpoint{2.987742in}{1.828851in}}%
\pgfpathlineto{\pgfqpoint{3.309820in}{1.848729in}}%
\pgfpathlineto{\pgfqpoint{3.631898in}{1.847218in}}%
\pgfpathlineto{\pgfqpoint{3.953976in}{1.845672in}}%
\pgfpathlineto{\pgfqpoint{4.276054in}{1.835775in}}%
\pgfpathlineto{\pgfqpoint{4.598132in}{1.816057in}}%
\pgfpathlineto{\pgfqpoint{4.920210in}{1.809519in}}%
\pgfpathlineto{\pgfqpoint{5.242288in}{1.810532in}}%
\pgfpathlineto{\pgfqpoint{5.564366in}{1.809994in}}%
\pgfusepath{stroke}%
\end{pgfscope}%
\begin{pgfscope}%
\pgfpathrectangle{\pgfqpoint{0.829820in}{0.819557in}}{\pgfqpoint{4.960000in}{3.696000in}}%
\pgfusepath{clip}%
\pgfsetbuttcap%
\pgfsetroundjoin%
\pgfsetlinewidth{1.204500pt}%
\definecolor{currentstroke}{rgb}{0.549020,0.549020,0.549020}%
\pgfsetstrokecolor{currentstroke}%
\pgfsetdash{{1.200000pt}{3.000000pt}{3.600000pt}{1.440000pt}}{0.000000pt}%
\pgfpathmoveto{\pgfqpoint{1.055275in}{1.131659in}}%
\pgfpathlineto{\pgfqpoint{1.377353in}{1.438854in}}%
\pgfpathlineto{\pgfqpoint{1.699431in}{1.533757in}}%
\pgfpathlineto{\pgfqpoint{2.021508in}{1.598913in}}%
\pgfpathlineto{\pgfqpoint{2.343586in}{1.636999in}}%
\pgfpathlineto{\pgfqpoint{2.665664in}{1.658829in}}%
\pgfpathlineto{\pgfqpoint{2.987742in}{1.692880in}}%
\pgfpathlineto{\pgfqpoint{3.309820in}{1.712313in}}%
\pgfpathlineto{\pgfqpoint{3.631898in}{1.713537in}}%
\pgfpathlineto{\pgfqpoint{3.953976in}{1.712173in}}%
\pgfpathlineto{\pgfqpoint{4.276054in}{1.702224in}}%
\pgfpathlineto{\pgfqpoint{4.598132in}{1.688540in}}%
\pgfpathlineto{\pgfqpoint{4.920210in}{1.683856in}}%
\pgfpathlineto{\pgfqpoint{5.242288in}{1.679899in}}%
\pgfpathlineto{\pgfqpoint{5.564366in}{1.682546in}}%
\pgfusepath{stroke}%
\end{pgfscope}%
\begin{pgfscope}%
\pgfsetrectcap%
\pgfsetmiterjoin%
\pgfsetlinewidth{1.254687pt}%
\definecolor{currentstroke}{rgb}{1.000000,1.000000,1.000000}%
\pgfsetstrokecolor{currentstroke}%
\pgfsetdash{}{0pt}%
\pgfpathmoveto{\pgfqpoint{0.829820in}{0.819557in}}%
\pgfpathlineto{\pgfqpoint{0.829820in}{4.515557in}}%
\pgfusepath{stroke}%
\end{pgfscope}%
\begin{pgfscope}%
\pgfsetrectcap%
\pgfsetmiterjoin%
\pgfsetlinewidth{1.254687pt}%
\definecolor{currentstroke}{rgb}{1.000000,1.000000,1.000000}%
\pgfsetstrokecolor{currentstroke}%
\pgfsetdash{}{0pt}%
\pgfpathmoveto{\pgfqpoint{5.789820in}{0.819557in}}%
\pgfpathlineto{\pgfqpoint{5.789820in}{4.515557in}}%
\pgfusepath{stroke}%
\end{pgfscope}%
\begin{pgfscope}%
\pgfsetrectcap%
\pgfsetmiterjoin%
\pgfsetlinewidth{1.254687pt}%
\definecolor{currentstroke}{rgb}{1.000000,1.000000,1.000000}%
\pgfsetstrokecolor{currentstroke}%
\pgfsetdash{}{0pt}%
\pgfpathmoveto{\pgfqpoint{0.829820in}{0.819557in}}%
\pgfpathlineto{\pgfqpoint{5.789820in}{0.819557in}}%
\pgfusepath{stroke}%
\end{pgfscope}%
\begin{pgfscope}%
\pgfsetrectcap%
\pgfsetmiterjoin%
\pgfsetlinewidth{1.254687pt}%
\definecolor{currentstroke}{rgb}{1.000000,1.000000,1.000000}%
\pgfsetstrokecolor{currentstroke}%
\pgfsetdash{}{0pt}%
\pgfpathmoveto{\pgfqpoint{0.829820in}{4.515557in}}%
\pgfpathlineto{\pgfqpoint{5.789820in}{4.515557in}}%
\pgfusepath{stroke}%
\end{pgfscope}%
\begin{pgfscope}%
\pgfsetbuttcap%
\pgfsetmiterjoin%
\definecolor{currentfill}{rgb}{0.917647,0.917647,0.949020}%
\pgfsetfillcolor{currentfill}%
\pgfsetlinewidth{0.803000pt}%
\definecolor{currentstroke}{rgb}{0.800000,0.800000,0.800000}%
\pgfsetstrokecolor{currentstroke}%
\pgfsetdash{}{0pt}%
\pgfpathmoveto{\pgfqpoint{4.116354in}{0.880668in}}%
\pgfpathlineto{\pgfqpoint{5.704265in}{0.880668in}}%
\pgfpathquadraticcurveto{\pgfqpoint{5.728709in}{0.880668in}}{\pgfqpoint{5.728709in}{0.905113in}}%
\pgfpathlineto{\pgfqpoint{5.728709in}{2.328046in}}%
\pgfpathquadraticcurveto{\pgfqpoint{5.728709in}{2.352490in}}{\pgfqpoint{5.704265in}{2.352490in}}%
\pgfpathlineto{\pgfqpoint{4.116354in}{2.352490in}}%
\pgfpathquadraticcurveto{\pgfqpoint{4.091909in}{2.352490in}}{\pgfqpoint{4.091909in}{2.328046in}}%
\pgfpathlineto{\pgfqpoint{4.091909in}{0.905113in}}%
\pgfpathquadraticcurveto{\pgfqpoint{4.091909in}{0.880668in}}{\pgfqpoint{4.116354in}{0.880668in}}%
\pgfpathclose%
\pgfusepath{stroke,fill}%
\end{pgfscope}%
\begin{pgfscope}%
\pgfsetroundcap%
\pgfsetroundjoin%
\pgfsetlinewidth{1.204500pt}%
\definecolor{currentstroke}{rgb}{0.298039,0.447059,0.690196}%
\pgfsetstrokecolor{currentstroke}%
\pgfsetdash{}{0pt}%
\pgfpathmoveto{\pgfqpoint{4.140798in}{2.253519in}}%
\pgfpathlineto{\pgfqpoint{4.385243in}{2.253519in}}%
\pgfusepath{stroke}%
\end{pgfscope}%
\begin{pgfscope}%
\definecolor{textcolor}{rgb}{0.150000,0.150000,0.150000}%
\pgfsetstrokecolor{textcolor}%
\pgfsetfillcolor{textcolor}%
\pgftext[x=4.483021in,y=2.210741in,left,base]{\color{textcolor}\sffamily\fontsize{8.800000}{10.560000}\selectfont BLAKE3}%
\end{pgfscope}%
\begin{pgfscope}%
\pgfsetbuttcap%
\pgfsetroundjoin%
\pgfsetlinewidth{1.204500pt}%
\definecolor{currentstroke}{rgb}{0.866667,0.517647,0.321569}%
\pgfsetstrokecolor{currentstroke}%
\pgfsetdash{{4.800000pt}{1.800000pt}}{0.000000pt}%
\pgfpathmoveto{\pgfqpoint{4.140798in}{2.074124in}}%
\pgfpathlineto{\pgfqpoint{4.385243in}{2.074124in}}%
\pgfusepath{stroke}%
\end{pgfscope}%
\begin{pgfscope}%
\definecolor{textcolor}{rgb}{0.150000,0.150000,0.150000}%
\pgfsetstrokecolor{textcolor}%
\pgfsetfillcolor{textcolor}%
\pgftext[x=4.483021in,y=2.031347in,left,base]{\color{textcolor}\sffamily\fontsize{8.800000}{10.560000}\selectfont BLAKE2s}%
\end{pgfscope}%
\begin{pgfscope}%
\pgfsetbuttcap%
\pgfsetroundjoin%
\pgfsetlinewidth{1.204500pt}%
\definecolor{currentstroke}{rgb}{0.333333,0.658824,0.407843}%
\pgfsetstrokecolor{currentstroke}%
\pgfsetdash{{1.200000pt}{1.200000pt}}{0.000000pt}%
\pgfpathmoveto{\pgfqpoint{4.140798in}{1.894730in}}%
\pgfpathlineto{\pgfqpoint{4.385243in}{1.894730in}}%
\pgfusepath{stroke}%
\end{pgfscope}%
\begin{pgfscope}%
\definecolor{textcolor}{rgb}{0.150000,0.150000,0.150000}%
\pgfsetstrokecolor{textcolor}%
\pgfsetfillcolor{textcolor}%
\pgftext[x=4.483021in,y=1.851952in,left,base]{\color{textcolor}\sffamily\fontsize{8.800000}{10.560000}\selectfont BLAKE2b}%
\end{pgfscope}%
\begin{pgfscope}%
\pgfsetbuttcap%
\pgfsetroundjoin%
\pgfsetlinewidth{1.204500pt}%
\definecolor{currentstroke}{rgb}{0.768627,0.305882,0.321569}%
\pgfsetstrokecolor{currentstroke}%
\pgfsetdash{{3.600000pt}{1.200000pt}{1.800000pt}{1.200000pt}}{0.000000pt}%
\pgfpathmoveto{\pgfqpoint{4.140798in}{1.715336in}}%
\pgfpathlineto{\pgfqpoint{4.385243in}{1.715336in}}%
\pgfusepath{stroke}%
\end{pgfscope}%
\begin{pgfscope}%
\definecolor{textcolor}{rgb}{0.150000,0.150000,0.150000}%
\pgfsetstrokecolor{textcolor}%
\pgfsetfillcolor{textcolor}%
\pgftext[x=4.483021in,y=1.672558in,left,base]{\color{textcolor}\sffamily\fontsize{8.800000}{10.560000}\selectfont BLAKE2sp}%
\end{pgfscope}%
\begin{pgfscope}%
\pgfsetbuttcap%
\pgfsetroundjoin%
\pgfsetlinewidth{1.204500pt}%
\definecolor{currentstroke}{rgb}{0.505882,0.447059,0.701961}%
\pgfsetstrokecolor{currentstroke}%
\pgfsetdash{{6.000000pt}{1.200000pt}{1.200000pt}{1.200000pt}}{0.000000pt}%
\pgfpathmoveto{\pgfqpoint{4.140798in}{1.535941in}}%
\pgfpathlineto{\pgfqpoint{4.385243in}{1.535941in}}%
\pgfusepath{stroke}%
\end{pgfscope}%
\begin{pgfscope}%
\definecolor{textcolor}{rgb}{0.150000,0.150000,0.150000}%
\pgfsetstrokecolor{textcolor}%
\pgfsetfillcolor{textcolor}%
\pgftext[x=4.483021in,y=1.493163in,left,base]{\color{textcolor}\sffamily\fontsize{8.800000}{10.560000}\selectfont BLAKE2bp}%
\end{pgfscope}%
\begin{pgfscope}%
\pgfsetbuttcap%
\pgfsetroundjoin%
\pgfsetlinewidth{1.204500pt}%
\definecolor{currentstroke}{rgb}{0.576471,0.470588,0.376471}%
\pgfsetstrokecolor{currentstroke}%
\pgfsetdash{{6.000000pt}{1.200000pt}{2.400000pt}{1.200000pt}{2.400000pt}{1.200000pt}}{0.000000pt}%
\pgfpathmoveto{\pgfqpoint{4.140798in}{1.356547in}}%
\pgfpathlineto{\pgfqpoint{4.385243in}{1.356547in}}%
\pgfusepath{stroke}%
\end{pgfscope}%
\begin{pgfscope}%
\definecolor{textcolor}{rgb}{0.150000,0.150000,0.150000}%
\pgfsetstrokecolor{textcolor}%
\pgfsetfillcolor{textcolor}%
\pgftext[x=4.483021in,y=1.313769in,left,base]{\color{textcolor}\sffamily\fontsize{8.800000}{10.560000}\selectfont OpenSSL SHA-256}%
\end{pgfscope}%
\begin{pgfscope}%
\pgfsetbuttcap%
\pgfsetroundjoin%
\pgfsetlinewidth{1.204500pt}%
\definecolor{currentstroke}{rgb}{0.854902,0.545098,0.764706}%
\pgfsetstrokecolor{currentstroke}%
\pgfsetdash{{2.400000pt}{2.400000pt}{3.600000pt}{1.800000pt}}{0.000000pt}%
\pgfpathmoveto{\pgfqpoint{4.140798in}{1.177152in}}%
\pgfpathlineto{\pgfqpoint{4.385243in}{1.177152in}}%
\pgfusepath{stroke}%
\end{pgfscope}%
\begin{pgfscope}%
\definecolor{textcolor}{rgb}{0.150000,0.150000,0.150000}%
\pgfsetstrokecolor{textcolor}%
\pgfsetfillcolor{textcolor}%
\pgftext[x=4.483021in,y=1.134374in,left,base]{\color{textcolor}\sffamily\fontsize{8.800000}{10.560000}\selectfont OpenSSL SHA-512}%
\end{pgfscope}%
\begin{pgfscope}%
\pgfsetbuttcap%
\pgfsetroundjoin%
\pgfsetlinewidth{1.204500pt}%
\definecolor{currentstroke}{rgb}{0.549020,0.549020,0.549020}%
\pgfsetstrokecolor{currentstroke}%
\pgfsetdash{{1.200000pt}{3.000000pt}{3.600000pt}{1.440000pt}}{0.000000pt}%
\pgfpathmoveto{\pgfqpoint{4.140798in}{0.997758in}}%
\pgfpathlineto{\pgfqpoint{4.385243in}{0.997758in}}%
\pgfusepath{stroke}%
\end{pgfscope}%
\begin{pgfscope}%
\definecolor{textcolor}{rgb}{0.150000,0.150000,0.150000}%
\pgfsetstrokecolor{textcolor}%
\pgfsetfillcolor{textcolor}%
\pgftext[x=4.483021in,y=0.954980in,left,base]{\color{textcolor}\sffamily\fontsize{8.800000}{10.560000}\selectfont OpenSSL SHA3-256}%
\end{pgfscope}%
\end{pgfpicture}%
\makeatother%
\endgroup%

\begin{tikzpicture}
\begin{axis}[
    width=0.9\textwidth,
	height=10cm,
	legend style={at={(0.98,0.98)}, anchor=north east, font=\small, cells={anchor=west}},
	axis x line=bottom,
	axis y line=left,
	cycle multi list={
		linestyles\nextlist
		blue,red,orange,green,cyan
	},
	%no markers,
	mark options={scale=0.5},
	every axis plot/.append style={thick},
    ymin=0,
    ymax=160,
    xmin=64,
    xmax=1048576,
	ylabel=Throughput (cpb),
	xlabel=Input Length,
	xmode=log,
	log basis x=2,
	y tick label style={%
		/pgf/number format/.cd,%
		scaled x ticks = false,%
		set thousands separator={},%
		fixed%
	},
	xticklabels={64 B,128 B,256 B,512 B,1 KiB,2 KiB,4 KiB,8 KiB,16 KiB,32 KiB,64 KiB,128 KiB,256 KiB,512 KiB,1 MiB},
	xtick={64, 128, 256, 512, 1024, 2048, 4096, 8192, 16384, 32768, 65536, 131072, 262144, 524288, 1048576},
	x tick label style={rotate=90,anchor=east}
]
\legend{BLAKE3, BLAKE2b, BLAKE2s, BLAKE2bp, BLAKE2sp, SHA-512, SHA-256, KangarooTwelve}
% blake3
\addplot coordinates {
(64, 30.628866117911816)
(80, 42.04692521249731)
(96, 35.020717855090616)
(112, 30.299413505135053)
(128, 26.544480266813707)
(160, 29.865536741247126)
(192, 25.056813771489093)
(224, 27.667474644404937)
(256, 24.36855251484833)
(320, 23.860916455991177)
(384, 23.679800183909347)
(448, 23.380818238679865)
(512, 23.228689180743032)
(640, 22.973864776577066)
(768, 22.837614891076242)
(896, 22.761951589380658)
(1024, 22.666595645822884)
(1280, 24.317429468289355)
(1536, 24.04317577518531)
(1792, 23.690457529773997)
(2048, 23.28618924575485)
(2560, 24.082909209390404)
(3072, 23.513636222947323)
(3584, 23.99244006187624)
(4096, 23.686215932351455)
(5120, 23.72900292582313)
(6144, 23.735075157042207)
(7168, 23.78711877654525)
(8192, 23.824276113245652)
(10240, 23.96898369696228)
(12288, 24.16761699478272)
(14336, 24.423814392843617)
(16384, 24.6050549276119)
(20480, 25.059969083506125)
(24576, 25.24124163835496)
(28672, 25.424524682030146)
(32768, 25.46698762686509)
(40960, 25.65445736071537)
(49152, 25.847885525237754)
(57344, 26.237595353692964)
(65536, 26.185441385469005)
(81920, 26.611262313822483)
(98304, 26.67578415924925)
(114688, 27.01702864979849)
(131072, 27.514401187099068)
(163840, 28.134074077666522)
(196608, 28.377266009863323)
(229376, 28.432094603948443)
(262144, 28.555346735969366)
(327680, 28.506977672091537)
(393216, 28.509812851173876)
(458752, 28.492855523664765)
(524288, 28.495064188682537)
(655360, 28.49089921188524)
(786432, 28.480958363592354)
(917504, 28.489371704331564)
(1048576, 28.472482593226896)
};
% blake2b
\addplot coordinates {
(64, 166.10963100697407)
(80, 133.31640308120845)
(96, 111.18947204286509)
(112, 95.55628814509829)
(128, 82.40847226409869)
(160, 127.03656693572952)
(192, 105.98039615458272)
(224, 91.11825031741677)
(256, 79.20528994767147)
(320, 93.88356978272455)
(384, 78.0201683622187)
(448, 88.62342416005193)
(512, 77.4300312601596)
(640, 76.99217389052572)
(768, 76.98949042326588)
(896, 76.82899737059927)
(1024, 76.51387728448864)
(1280, 76.47155991874841)
(1536, 76.3142968294614)
(1792, 76.43693453178106)
(2048, 76.19091797266704)
(2560, 76.10373607692509)
(3072, 76.07864536402917)
(3584, 76.23414081075077)
(4096, 76.13544292221644)
(5120, 76.10111846899474)
(6144, 76.01511124471239)
(7168, 76.19776292083166)
(8192, 76.07164224129968)
(10240, 76.20220294590821)
(12288, 76.3918998756219)
(14336, 76.90824839541023)
(16384, 76.7313213749645)
(20480, 77.2784716449082)
(24576, 77.39486272985609)
(28672, 77.65051661628237)
(32768, 77.62554784983075)
(40960, 77.75533312788048)
(49152, 77.96612128000929)
(57344, 78.20547520648847)
(65536, 78.30439427919924)
(81920, 79.0390489395703)
(98304, 79.40344720396148)
(114688, 79.71726871274574)
(131072, 80.2135742803083)
(163840, 80.56165251252344)
(196608, 80.47225589504723)
(229376, 80.5929193883054)
(262144, 80.6522437169574)
(327680, 80.61193034519506)
(393216, 80.66401965612991)
(458752, 80.61537754572228)
(524288, 80.60694217706124)
(655360, 80.62954662777919)
(786432, 80.63080115114842)
(917504, 80.63395311086575)
(1048576, 80.61723481571902)
};
% blake2s
\addplot coordinates {
(64, 37.82043598586084)
(80, 57.90058660742277)
(96, 48.02111829121259)
(112, 41.3392194906987)
(128, 34.642494919786806)
(160, 41.216954116800544)
(192, 33.52106681816118)
(224, 38.351706608936134)
(256, 32.927048686005584)
(320, 32.60754620264363)
(384, 32.39909214101215)
(448, 32.12699249738638)
(512, 32.07102933118352)
(640, 31.882609057724615)
(768, 31.787368516366787)
(896, 31.697879784341325)
(1024, 31.624362338861623)
(1280, 31.57125369606177)
(1536, 31.447611880740975)
(1792, 31.484726011088917)
(2048, 31.392152585671823)
(2560, 31.41485665162383)
(3072, 31.396880406907297)
(3584, 31.374351970569858)
(4096, 31.40261394083015)
(5120, 31.259663426123225)
(6144, 31.304243829503342)
(7168, 31.313100285788174)
(8192, 31.32013828941087)
(10240, 31.49608426188788)
(12288, 31.442175310566853)
(14336, 31.510621336249933)
(16384, 31.731498538908376)
(20480, 32.29815699631195)
(24576, 32.35717126785338)
(28672, 32.47252182030351)
(32768, 32.68494403472129)
(40960, 32.727846898681676)
(49152, 32.85419123400104)
(57344, 32.85983148689304)
(65536, 32.96130295410878)
(81920, 33.19913023875922)
(98304, 33.403357153700334)
(114688, 33.96948941523164)
(131072, 34.70444653332489)
(163840, 35.08218377179551)
(196608, 35.40830965696312)
(229376, 35.644176612214096)
(262144, 35.68130767389328)
(327680, 35.69737357770137)
(393216, 35.654394258213806)
(458752, 35.648339215050136)
(524288, 35.683690828447666)
(655360, 35.64717344563382)
(786432, 35.664910536538166)
(917504, 35.64367700441932)
(1048576, 35.65955336730132)
};
% blake2bp
\addplot coordinates {
(64, 982.6253155854007)
(80, 788.2864198338443)
(96, 655.1954669393594)
(112, 562.6321525186568)
(128, 490.63216796875)
(160, 392.8705992281014)
(192, 327.0002571523185)
(224, 280.69519830018055)
(256, 244.82016830009147)
(320, 196.50708493193068)
(384, 163.8717454482884)
(448, 139.99275993315115)
(512, 122.46072272299452)
(640, 112.98368434689615)
(768, 106.95702886166558)
(896, 102.23668468386018)
(1024, 99.17452790310514)
(1280, 94.3330982920587)
(1536, 91.36229104900929)
(1792, 89.1036608339767)
(2048, 87.73441345662813)
(2560, 85.121317235989)
(3072, 83.77712534894756)
(3584, 82.60859693848099)
(4096, 81.7859666614618)
(5120, 80.69586316964835)
(6144, 79.88945898857611)
(7168, 79.36371684915734)
(8192, 79.06716012825338)
(10240, 78.69525534937472)
(12288, 78.34986554130357)
(14336, 78.61745305762669)
(16384, 78.52848546279134)
(20480, 78.69346104356966)
(24576, 78.62049897167743)
(28672, 78.9421691907674)
(32768, 78.67412195935424)
(40960, 78.61605427512711)
(49152, 78.694215837173)
(57344, 78.81366011394583)
(65536, 78.90612942647843)
(81920, 79.28526959795845)
(98304, 79.79014017957354)
(114688, 80.40488714125567)
(131072, 80.92809623085768)
(163840, 81.45187699373092)
(196608, 81.77836824237005)
(229376, 81.68254041844452)
(262144, 81.67593544404335)
(327680, 81.60474832483885)
(393216, 81.54835478718624)
(458752, 81.49483156123974)
(524288, 81.46802131545161)
(655360, 81.41900266194276)
(786432, 81.35182164550679)
(917504, 81.36381542288999)
(1048576, 81.34996621499303)
};
% blake2sp
\addplot coordinates {
(64, 436.89498061210037)
(80, 352.2491640746057)
(96, 293.10372461725336)
(112, 251.37634687921275)
(128, 218.22196814889043)
(160, 175.8779829016017)
(192, 145.42018248622128)
(224, 125.20056766497545)
(256, 108.84909300637385)
(320, 87.63978773084865)
(384, 72.58230615179056)
(448, 61.84599001379682)
(512, 54.111196336605445)
(640, 49.503470623268996)
(768, 46.536763597918295)
(896, 44.444221885225566)
(1024, 42.73005059294105)
(1280, 40.46758839035023)
(1536, 38.936275940704526)
(1792, 37.81841622586766)
(2048, 37.002168730513695)
(2560, 35.807787111290416)
(3072, 35.10379756500603)
(3584, 34.65187423071145)
(4096, 34.10324052285563)
(5120, 33.540176181957634)
(6144, 33.16914371280587)
(7168, 32.908427062579825)
(8192, 32.67219149231435)
(10240, 32.57462011521523)
(12288, 32.50231700505055)
(14336, 32.566403720683404)
(16384, 32.631624206209246)
(20480, 32.98610579395816)
(24576, 33.189216156464354)
(28672, 33.417071690168704)
(32768, 33.504580561571785)
(40960, 33.578444517987684)
(49152, 33.716066814979)
(57344, 33.814333914393764)
(65536, 33.99830043751573)
(81920, 34.33632459735377)
(98304, 34.97587627501272)
(114688, 36.00483704093893)
(131072, 36.29014139986327)
(163840, 36.72459438261176)
(196608, 37.2751465911512)
(229376, 37.310827453635184)
(262144, 37.409209762196674)
(327680, 37.42167597214834)
(393216, 37.4116376426174)
(458752, 37.3801409960235)
(524288, 37.36729139342378)
(655360, 37.324853326079584)
(786432, 37.28470643898635)
(917504, 37.27781739974022)
(1048576, 37.25260767845806)
};
% sha512
\addplot coordinates {
(64, 210.73354353748348)
(80, 169.24605441091032)
(96, 141.31396555090745)
(112, 204.0797256362763)
(128, 176.02016418036393)
(160, 142.2174483430619)
(192, 118.47455035942028)
(224, 102.14105402316402)
(256, 124.98037137464723)
(320, 100.27918684335032)
(384, 106.5001679581671)
(448, 91.3891935291277)
(512, 97.74357169299313)
(640, 92.47181047867463)
(768, 89.07211508746491)
(896, 86.5295946942032)
(1024, 84.6748012810041)
(1280, 81.95667548211911)
(1536, 80.3774733793998)
(1792, 79.04918082232616)
(2048, 78.25763083368739)
(2560, 76.6883212663477)
(3072, 76.04004744089731)
(3584, 75.35302477952352)
(4096, 75.05076155261027)
(5120, 74.42131323176045)
(6144, 73.98857573018262)
(7168, 73.55544081564089)
(8192, 73.49595108611732)
(10240, 73.26088603586338)
(12288, 73.27184231262372)
(14336, 73.41620166975304)
(16384, 73.25265381977796)
(20480, 73.36149509727238)
(24576, 73.60337457531244)
(28672, 73.56101086701777)
(32768, 73.56345341234417)
(40960, 73.88006808871697)
(49152, 73.72033084872956)
(57344, 73.89989831908132)
(65536, 74.041165408764)
(81920, 74.47571871529286)
(98304, 74.90712965342783)
(114688, 75.52360350637355)
(131072, 76.11578024344306)
(163840, 76.33485466146604)
(196608, 76.40279729983787)
(229376, 76.40049398566434)
(262144, 76.32809136539649)
(327680, 76.29739993438113)
(393216, 76.28307610843764)
(458752, 76.29663194834146)
(524288, 76.24182712073514)
(655360, 76.26116708004673)
(786432, 76.22860434545179)
(917504, 76.24048035206565)
(1048576, 76.2153062022178)
};
% sha256
\addplot coordinates {
(64, 128.90129196701326)
(80, 106.44180042823184)
(96, 89.07959888218943)
(112, 76.55340099629457)
(128, 80.54392299577876)
(160, 66.6487349637949)
(192, 64.55770815075674)
(224, 57.15559458272206)
(256, 56.3337233993212)
(320, 51.525978714872416)
(384, 48.45552777209484)
(448, 47.01790524205502)
(512, 45.12993987177213)
(640, 41.96590514592378)
(768, 40.42941439971015)
(896, 39.22627249079312)
(1024, 38.33387068241505)
(1280, 37.07904904644916)
(1536, 36.328053020038865)
(1792, 35.762260599730475)
(2048, 35.287252260015784)
(2560, 34.737045050794144)
(3072, 34.28545764294247)
(3584, 34.06980265906222)
(4096, 33.91988599269591)
(5120, 33.61921420665854)
(6144, 33.568871314163246)
(7168, 33.45952429347912)
(8192, 33.27156784930256)
(10240, 33.23615042275858)
(12288, 33.23522894371993)
(14336, 33.25750153878327)
(16384, 33.34546459551451)
(20480, 33.67316529911358)
(24576, 33.74759956995365)
(28672, 33.87673099862416)
(32768, 33.962318451121654)
(40960, 33.97917516640672)
(49152, 33.97078065861951)
(57344, 33.99600961098388)
(65536, 34.167124560685785)
(81920, 34.28652695969843)
(98304, 34.7827094580386)
(114688, 35.13695362030102)
(131072, 35.662827548025206)
(163840, 36.47393724902257)
(196608, 36.60821836991401)
(229376, 36.59828790652875)
(262144, 36.58884055403569)
(327680, 36.57106028757254)
(393216, 36.56526139759312)
(458752, 36.52805937195758)
(524288, 36.52801510283351)
(655360, 36.517187449003586)
(786432, 36.5012341971003)
(917504, 36.48939868038819)
(1048576, 36.51364289598732)
};
% kangarootwelve
\addplot coordinates {
(64, 135.35694969819727)
(80, 110.03978892066)
(96, 93.2906764381309)
(112, 80.89381878603491)
(128, 71.6456343745382)
(160, 58.83023545866846)
(192, 85.45864453600365)
(224, 74.43355407676962)
(256, 65.85965630991659)
(320, 54.2307903250164)
(384, 63.88399551442107)
(448, 55.811923878222046)
(512, 62.8489926929847)
(640, 51.916752370726805)
(768, 53.10948164576252)
(896, 54.263533041846486)
(1024, 54.83034626078046)
(1280, 50.77022795612555)
(1536, 52.08277116061922)
(1792, 49.580337835497645)
(2048, 51.15357378884315)
(2560, 50.23083182387298)
(3072, 49.63882527489739)
(3584, 49.163796557814926)
(4096, 48.8215832010469)
(5120, 48.49823101771465)
(6144, 48.17003124232738)
(7168, 48.04788952575129)
(8192, 49.782714011427814)
(10240, 49.30049444902075)
(12288, 49.02000472142305)
(14336, 48.869817879515416)
(16384, 49.38819811311803)
(20480, 49.23929542595847)
(24576, 49.55499213509053)
(28672, 49.42927023976239)
(32768, 49.59251386572274)
(40960, 49.70598302553408)
(49152, 49.74295254028195)
(57344, 50.091115757806634)
(65536, 50.138867076469595)
(81920, 50.412480293915436)
(98304, 51.08615707587623)
(114688, 51.52125783016282)
(131072, 51.83176694493148)
(163840, 52.17928199147748)
(196608, 52.40431609013052)
(229376, 52.46032051812981)
(262144, 52.469957384301445)
(327680, 52.406942252029324)
(393216, 52.41612302359709)
(458752, 52.41837240559161)
(524288, 52.41830244091671)
(655360, 52.38320639093105)
(786432, 52.404504159264334)
(917504, 52.38675568129792)
(1048576, 52.366879964816945)
};
\end{axis}
\end{tikzpicture}

\caption{Throughput for single-threaded BLAKE3 and other hash functions on a
    Rasperry Pi Zero, measured in cycles per byte. This is a 32-bit ARM1176
    processor. Lower is faster.}%
\label{fig:rpizero}
\end{figure}

Figure~\ref{fig:rpizero} shows the throughput of single-threaded BLAKE3 on a
smaller embedded platform. This benchmark ran on a Raspberry Pi Zero with a
32-bit ARM1176 processor. Here, the biggest difference between algorithms is
whether they use 32-bit or 64-bit words. As noted above, BLAKE3 and BLAKE2s
have similar performance profiles here, because their compression functions are
closely related, and because there is no multi-core or SIMD parallelism for
BLAKE3 to exploit.

\begin{figure}[h]
\centering
\begin{tikzpicture}
\begin{axis}[
    width=0.9\textwidth,
	height=10cm,
	legend style={at={(0.02,0.98)}, anchor=north west, font=\small, cells={anchor=west}},
	axis x line=bottom,
	axis y line=left,
	cycle multi list={
		linestyles\nextlist
		blue,red,orange,green,cyan
	},
	%no markers,
	mark options={scale=0.5},
	every axis plot/.append style={thick},
    ymin=0,
    ymax=100,
    xmin=32768,
    xmax=1073741824,
	ylabel=Throughput (GiB/s),
	xlabel=Input Length,
	xmode=log,
	log basis x=2,
	y tick label style={%
		/pgf/number format/.cd,%
		scaled x ticks = false,%
		set thousands separator={},%
		fixed%
	},
	xticklabels={32 KiB,64 KiB,128 KiB,256 KiB,512 KiB,1 MiB,2 MiB,4 MiB,8 MiB,16 MiB,32 MiB,64 MiB,128 MiB,256 MiB,512 MiB,1 GiB},
	xtick={32768, 65536, 131072, 262144, 524288, 1048576, 2097152, 4194304, 8388608, 16777216, 33554432, 67108864, 134217728, 268435456, 536870912, 1073741824},
	x tick label style={rotate=90,anchor=east}
]
\legend{48 threads,32 threads,16 threads,8 threads,4 threads,2 threads,1 thread}
% threads_48
\addplot coordinates {
(32768, 2.887027968482478)
(65536, 4.071455979594155)
(131072, 6.528819781738289)
(262144, 10.513502207359979)
(524288, 17.394757194898673)
(1048576, 27.430977269670468)
(2097152, 40.605907967720675)
(4194304, 57.23719684917102)
(8388608, 71.49745198426369)
(16777216, 88.43340036974317)
(33554432, 96.91074384165438)
(67108864, 93.0809238665907)
(134217728, 89.03923134997731)
(268435456, 88.68239270785789)
(536870912, 89.30977250246572)
(1073741824, 89.32288760218576)
};
% threads_32
\addplot coordinates {
(32768, 3.8734902265344564)
(65536, 5.869527452900506)
(131072, 9.03469820524766)
(262144, 14.759262437473932)
(524288, 23.43064282274582)
(1048576, 33.89170167295339)
(2097152, 47.45466230184613)
(4194304, 61.8180083018402)
(8388608, 77.63156189285043)
(16777216, 89.64899373144745)
(33554432, 93.59169396453068)
(67108864, 85.24647777400084)
(134217728, 78.85868696498126)
(268435456, 77.92419800960157)
(536870912, 77.86955485200878)
(1073741824, 78.26540078562837)
};
% threads_16
\addplot coordinates {
(32768, 5.361086229899734)
(65536, 8.714329052489145)
(131072, 13.77468767301237)
(262144, 20.582716544748095)
(524288, 29.883337215934553)
(1048576, 39.128401240913036)
(2097152, 50.01581572192307)
(4194304, 58.84215102342774)
(8388608, 66.07408040430856)
(16777216, 70.22537828011971)
(33554432, 72.44299230672223)
(67108864, 58.88645276186061)
(134217728, 54.29159516716906)
(268435456, 53.15674304363236)
(536870912, 52.68269837826563)
(1073741824, 51.792352716152855)
};
% threads_08
\addplot coordinates {
(32768, 6.717451391246451)
(65536, 10.269888841196435)
(131072, 14.356475752047873)
(262144, 18.98395417446593)
(524288, 25.25483454900287)
(1048576, 30.47555605437972)
(2097152, 33.861316317840114)
(4194304, 36.02230437063406)
(8388608, 37.66848550769804)
(16777216, 38.78348673260553)
(33554432, 37.18342582425126)
(67108864, 30.46166146497173)
(134217728, 28.217924832976998)
(268435456, 27.471751223695524)
(536870912, 26.441196658801605)
(1073741824, 25.034595606294772)
};
% threads_04
\addplot coordinates {
(32768, 7.586757745178237)
(65536, 10.475700845877443)
(131072, 12.906672489264153)
(262144, 15.827964135793277)
(524288, 18.457251468815976)
(1048576, 20.099323091341002)
(2097152, 20.08160341331945)
(4194304, 20.207696083351)
(8388608, 20.75693770021205)
(16777216, 21.0118307300413)
(33554432, 19.10377073874376)
(67108864, 15.996760584144885)
(134217728, 14.81055786406246)
(268435456, 13.882877495452354)
(536870912, 12.959815946700441)
(1073741824, 12.425577137656836)
};
% threads_02
\addplot coordinates {
(32768, 7.913858698146113)
(65536, 9.65148224915338)
(131072, 10.861641759231814)
(262144, 11.60250618313749)
(524288, 11.954477397808196)
(1048576, 12.167835929647827)
(2097152, 11.69789307004965)
(4194304, 10.791957729983883)
(8388608, 11.155465218040767)
(16777216, 11.104563817755597)
(33554432, 10.402082025595815)
(67108864, 8.805432689093701)
(134217728, 7.808860246304982)
(268435456, 7.177518496203779)
(536870912, 6.9091992262389565)
(1073741824, 6.781625476884904)
};
% threads_01
\addplot coordinates {
(32768, 6.1843242451207106)
(65536, 6.151595515127176)
(131072, 6.220617561379796)
(262144, 6.2541492627479025)
(524288, 6.250371042276127)
(1048576, 5.810744011932857)
(2097152, 5.554140851358214)
(4194304, 5.444389445236837)
(8388608, 5.603722023882172)
(16777216, 5.602913915758918)
(33554432, 5.203544788013157)
(67108864, 4.303747663922745)
(134217728, 4.01864957824464)
(268435456, 3.8607512936406607)
(536870912, 3.818848296625481)
(1073741824, 3.770164351726395)
};
\end{axis}
\end{tikzpicture}

    \caption{Throughput for multi-threaded BLAKE3 on the same c5.metal instance
    as in Figure~\ref{fig:avx512}, measured in GiB/s, varying the number of
    threads. This machine has 48 physical cores. Higher is faster.}%
\label{fig:threads}
\end{figure}

Figure~\ref{fig:threads} shows the throughput of multi-threaded BLAKE3. This
benchmark ran on the same c5.metal instance as in Figure~\ref{fig:avx512},
which has 48 physical cores. This implementation uses the Rust Rayon library to
manage threading, with the recursive divide-and-conquer approach described in
\S\ref{sec:multithreading}. Below 128~KiB of input, the overhead of threading
is high, and increasing the number of threads causes a performance drop. For
longer inputs, throughput scales almost linearly with the number of threads, up
to 16 threads. Above 16 threads, throughput levels off due to memory bandwidth
effects.

Figures~\ref{fig:avx512} and~\ref{fig:rpizero} also highlight that BLAKE3
performs well for very short inputs, at or below its 64-byte block size. Very
short inputs get padded up to a full block, and algorithms with smaller block
sizes benefit from compressing less padding. The fixed-interleaved tree
structures of BLAKE2bp and BLAKE2sp are especially costly when the input is
short, because they always compress a parent node and a fixed number of leaves.
The advantage of a fixed-interleaved tree structure comes at medium input
lengths, where BLAKE3 does not yet have enough chunks to operate in parallel.
This is the regime around 1--4\,KiB in Figure~\ref{fig:avx512}, where BLAKE2bp
and BLAKE2sp pull ahead.

The SHA\=/2 implementations used in these benchmarks are from OpenSSL, and the
KangarooTwelve implementations are from the eXtended Keccak Code Package
(compiled with \texttt{gcc -Os} for the ARM1176).

\section{Security}\label{sec:security}

\subsection{Security Goals}\label{sec:goals}

BLAKE3 targets $128$-bit security for all of its security goals. That
is, $128$-bit security against (second-)preimage, collision, or
differentiability attacks. 
The rationale for targeting $128$-bit security can be found
in~\cite[\S2]{TMC}.

The key length is nevertheless $256$ bits, as an extra defense layer
against weakly chosen keys, key material leaks, and potentially
multi-target attacks.

%Users who need $256$-bit security can remain using BLAKE2.

\subsection{Compression Function}\label{sec:compressindiff}

The BLAKE3 compression function is identical to that of BLAKE2s, with
the exception that:

\begin{itemize}
  \item BLAKE2's $f_0$ and $f_1$ ``flags'' are replaced by the $b$ and $d$ inputs, which
  encode the functional context of the compression function instance.
  \item Finalization returns $16$ words instead of $8$, and only applies feed-forward on the second half of the output.
\end{itemize}

The latter change causes the BLAKE3 function to no longer be indifferentiable, unlike its predecessor~\cite{DBLP:journals/tosc/LuykxMN16}. Indeed, an attacker with oracle access to $C(\dots)$ and $E(\dots)$ that obtains $y = C(x, m, b, d)$ can compute $(x, b, d) = E^{-1}(m, (y_l, y_h) \oplus (y_h, 0) \oplus (0, x_l))$; a simulator with access to a random oracle cannot perform the same inversion, and as such this scheme---much like Davies-Meyer---is not indifferentiable from a random oracle. This causes some surmountable complications in the next section.

We do note that the truncated compression function used in every compression except the root nodes is indifferentiable. The simulator and proof are very similar to those in~\cite{DBLP:journals/tosc/LuykxMN16}, with the only change being the removal of the feed-forward, which has no effect on security when truncating the output.\footnote{A recent work~\cite{DBLP:conf/asiacrypt/ChoiLL19} improves upon~\cite{DBLP:conf/fse/DodisRRS09,DBLP:journals/tosc/LuykxMN16} with a tighter bound for the indifferentiability of truncated permutations. A similar approach would work for BLAKE2 or truncated BLAKE3, but the benefits for our parameters are negligible.}

\subsection{Mode of Operation}\label{sec:mode}

One important aspect of any hash mode, parallel or otherwise, is its \emph{soundness}. Several works~\cite{DBLP:conf/fse/DodisRRS09,DBLP:journals/iacr/DaemenDA11,DBLP:journals/ijisec/BertoniDPA14,DBLP:journals/tosc/DaemenMA18} have studied the requirements for the mode of operation of a hash function such that, when instantiated with an ideal compression function, block cipher, or permutation, it remains indistinguishable from a random oracle. We adopt the requirements from Daemen et al.~\cite{DBLP:journals/tosc/DaemenMA18}, which are:
\begin{itemize}
  \item Subtree-freeness, i.e, no valid tree may be a subtree of another;
  \item Radical-decodability, i.e., preventing collisions by ambiguity of the tree shape;
  \item Message-completeness, i.e., the entire message can be recovered from the inputs to the primitive in question.
\end{itemize}

\paragraph{Subtree-freeness}{
  Subtree-freeness ensures that generalized length-extension attacks, like the ones that plagued Merkle-Damg{\aa}rd, cannot happen.
  To ensure subtree-freeness, BLAKE3 defines the \flag{ROOT} flag, which is only set on the last compression output. Thus, for any valid tree, a valid subtree must have the same root node. There are two cases to consider here:
  \begin{enumerate}
    \item The \flag{ROOT} flag is combined with \flag{CHUNK_END}; in this case we are dealing with a single-chunk message, and the root nodes must be the same, as well as its parents, grandparents, etc, until we reach a \flag{CHUNK_START} node. Because both subtrees must start with \flag{CHUNK_START}, we conclude that both trees must be the same.
    \item The \flag{ROOT} flag is applied to a parent node. Here the root node necessarily has two parents, each of which must be equal, recursing until they hit the same set of leaves marked by the \flag{CHUNK_END} flag.
  \end{enumerate}
}
\paragraph{Radical-decodability}{
  For radical-decodability, we can define the subset of final nodes. For each final node, one can define a function $\text{radical}()$ that returns a CV position for any final node. Here we have, again, two cases:
  \begin{enumerate}
    \item The final node has the \flag{CHUNK_END} flag set; here the radical is in the CV bits.
    \item The final node is a parent node; here the CV comes in the message bits.
  \end{enumerate}
}
\paragraph{Message-completeness}{
  The entire input can be recovered from the message bits input to the chunks.
}

If only considering 256-bit outputs or shorter, by \cite[Theorem~1]{DBLP:journals/tosc/DaemenMA18}, for a distinguisher $\mathcal{D}$ of total complexity $q$ we have
\begin{align}
  \mathsf{Adv}_{BLAKE3}^{\text{diff}}(\mathcal{D}) \le & ~\frac{\binom{q}{2}}{2^{256}} +\label{eq:thm1} \\
  & ~\frac{\binom{q}{2}}{2^{512}} + \frac{\binom{q}{2}}{2^{256}} + \frac{q}{2^{128}} \label{eq:thm2} \,.
\end{align}
Here (\ref{eq:thm1}) corresponds to the contribution of \cite[Theorem~1]{DBLP:journals/tosc/DaemenMA18} and (\ref{eq:thm2}) corresponds to the indifferentiability of the truncated compression function~\cite{DBLP:journals/tosc/LuykxMN16}.

For the full-length output, we do not have a truncated ideal cipher, but instead use a feed-forward to prevent inversion. We remark, however, that the truncation requirement in \cite[Theorem~3]{DBLP:journals/tosc/DaemenMA18} is in place purely to prevent the distinguisher from inverting an output; as such, the feed-forward serves the same purpose as truncation, and the result still follows as long as the other soundness requirements are in place.

\subsection{Cryptanalysis}\label{sec:cryptanalysis}

\begin{table}[t]
  \centering
  \caption{Summary of cryptanalysis on BLAKE and BLAKE2.}%
  \label{tab:cryptanalysis}
  \begin{tabular}{lcccc}
    \toprule
    Primitive      & Type         & Rounds & Complexity & Reference \\
    \midrule
    BLAKE-256 k.p. & Boomerang    & 7      & $2^{44}$    & \cite{DBLP:journals/iet-ifs/BaiYWW15,DBLP:conf/cisc/Hao14} \\
    BLAKE-256 k.p. & Boomerang    & 8      & $2^{198}$    & \cite{DBLP:conf/cisc/Hao14} \\
    BLAKE-256 k.p. & Differential & 4      & $2^{192}$    & \cite{DK11} \\
    BLAKE-256 k.p. & Differential & 6      & $2^{456}$    & \cite{DK11} \\
    BLAKE-256 k.p. & Impossible Differential & 6.5      & --    & \cite{DBLP:conf/ctrsa/0001KNWW14} \\
    BLAKE-256 c.f. & Boomerang    & 7      & $2^{242}$    & \cite{DBLP:conf/fse/BiryukovNR11} \\
    BLAKE-256 c.f. & Near collision (152/256) & 4      & $2^{21}$    & \cite{DBLP:conf/cans/SuWWD10} \\
    BLAKE-256 c.f. & Near collision (232/256) & 4      & $2^{56}$    & \cite{DBLP:conf/fse/AumassonGKMM10} \\
    BLAKE-256 c.f. & Preimage & 2.5      & $2^{241}$    & \cite{DBLP:journals/iacr/JiL09} \\
    BLAKE-256 c.f. & Pseudo-preimage & 6.75 & $2^{253.9}$    & \cite{DBLP:conf/crypto/EspitauFK15} \\
    \midrule
    BLAKE2s k.p.   & Boomerang    & 7.5    & $2^{184}$    & \cite{DBLP:conf/cisc/Hao14} \\
    BLAKE2s k.p.   & Impossible Differential & 6.5      & --    & \cite{DBLP:conf/ctrsa/0001KNWW14} \\
    BLAKE2s k.p.   & Rotational   & 4    & $2^{489}$    & \cite{DBLP:conf/ctrsa/0001KNWW14,DBLP:conf/fse/KhovratovichNPS15} \\
    BLAKE2s c.f. & Boomerang & 5 & $\approx 2^{86}$    & \cite{DBLP:conf/fse/BiryukovNR11,DBLP:conf/ctrsa/0001KNWW14} \\
    BLAKE2s c.f. & Pseudo-preimage & 6.75 & $2^{253.8}$    & \cite{DBLP:conf/crypto/EspitauFK15} \\
    \bottomrule
  \end{tabular}
\end{table}

The round reduction from 10 to 7 rounds is based on existent cryptanalysis, along with novel research.
High confidence in the security of 7 rounds follows from cryptanalysis
efforts over 10 years, first on
BLAKE~\cite{DBLP:conf/cans/SuWWD10,DBLP:journals/ipl/VidaliNP10,DBLP:journals/iet-ifs/BaiYWW15,DBLP:conf/fse/AumassonGKMM10,DBLP:journals/iacr/JiL09,DK11,DBLP:conf/fse/BiryukovNR11}
and then on BLAKE2~\cite{DBLP:conf/ctrsa/0001KNWW14,DBLP:conf/cisc/Hao14,DBLP:conf/crypto/EspitauFK15}. Table~\ref{tab:cryptanalysis} summarizes the current state of the art in BLAKE(2) cryptanalysis.

Furthermore, our own searches for good\footnote{Optimal under certain assumptions, which are not entirely correct, cf. \cite{cryptoeprint:2013:328}. Nevertheless we find these results useful as a first approximation.} differential trails, shown in Table~\ref{tab:trails}, show that by restricting input differences the initialization of BLAKE2/3 makes things harder for the attacker, with probabilities quickly becoming very low. This was also observed in~\cite[\S7]{DBLP:conf/ctrsa/0001KNWW14}. 

The boomerang attacks of, e.g., \cite{DBLP:conf/cisc/Hao14,DBLP:conf/fse/BiryukovNR11,DBLP:journals/iet-ifs/BaiYWW15} exploit the high probabilities on the first few rounds to connect two trails and get through a few more rounds. This is a usual trick to get around the very quick lowering of probabilities in ARX constructs, along with differential-linear attacks. But such attacks do not present a threat to BLAKE3, considering its $128$-bit security target, and input constraints on the compression function (i.e., the IV constants). \alert{This paragraph is overly informal and doesn't really say much. Could be removed.}

The more general survey and analysis in~\cite{TMC} also pleads in favor
of reduced round values, arguing that BLAKE2s and BLAKE2b would be as
safe with 7 and 8 rounds, respectively.

\begin{table}[t]
  \centering
  \caption{Best differential trail probabilities for increasing round numbers of the compression functions of BLAKE-256, BLAKE2s, and BLAKE3, respecting their constraints on inputs to the keyed permutation. Probabilities marked with ${}^\ast$ indicate rotationally symmetric differences were sought exclusively.}%
  \label{tab:trails}
  \begin{tabular}{lccccccc}
    \toprule
    Function & 0.5   & 1     & 1.5   & 2     & 2.5   & 3     & 3.5 \\ \midrule
    BLAKE-256 c.f. & $2^{-0}$ & $2^{-0}$ & $2^{-0}$ & $2^{-1}$ & $2^{-6}$ & $2^{-7}$ & $2^{-38}$  \\
    BLAKE\{2s,3\} c.f.   &  $2^{-0}$  & $2^{-0}$ &  $2^{-0}$  & $2^{-1}$ & $2^{-32}$   & $\left[2^{-88}, 2^{-48}\right)$ & -  \\
    BLAKE\{-256,2s,3\} (hash) & $2^{-0}$ & $2^{-1}$ & $2^{-32}$ & ${\ge 2^{-190}}^\ast$ & - & - & -  \\ % 2^{-190} with 8-symmetric $\Delta$, $2^{-230}$ with 4-symmetric $\Delta$ m10 = 0x88888888
    BLAKE-256 c.f. (fixed $m$) & $2^{-0}$ & $2^{-2}$ & $2^{-12}$ & $2^{-39}$ & - & - & - \\
    BLAKE\{2s,3\} c.f. (fixed $m$) & $2^{-0}$ & $2^{-7}$ & $2^{-32}$ & ${\ge 2^{-161}}^\ast$ & - & - & - \\ % 2^{-161} with 8-symmetric $\Delta$, $2^{-161}$ with 4-symmetric $\Delta$
    \bottomrule
  \end{tabular}
\end{table}

\section{Implementation}\label{sec:implementation}

\subsection{Incremental Hashing}\label{sec:incremental}

An incremental implementation of BLAKE3 has two major components: the state of
the current chunk and a stack of subtree chaining values (the ``CV stack'').
The chunk state is structurally similar to an incremental implementation of
BLAKE2b or BLAKE2s, and it will be familiar to implementers of other hash
functions. The CV stack is less familiar. Simplifying the management of the CV
stack is especially important for a clear and correct implementation of BLAKE3.
This section describes how this is done in the
\href{https://github.com/veorq/BLAKE3/blob/master/reference_impl/reference_impl.rs}{reference
implementation}, as an aid to implementers.
% TODO: repoint this to BLAKE3-team and a specific tag

\subsubsection{Chunk State}\label{sec:chunkstate}

The chunk state contains the 32-byte CV of the previous block and a 64-byte
input buffer for the next block, and typically also the compression function
parameters $t$ and $d$. Input bytes from the caller are copied into the buffer
until it is full. Then the buffer is compressed together with the previous CV,
using the truncated compression function. The output CV overwrites the previous
CV, and the buffer is cleared. An important detail here is that the last block
of a chunk is compressed differently, setting the \flag{CHUNK_END} flag and
possibly the \flag{ROOT} flag. In an incremental setting, any block could
potentially be the last, until the caller has supplied enough input to place at
least 1 byte in the block after. For that reason, the chunk state waits to
compress the next block until both the buffer is full and the caller has
supplied more input. Note that the CV stack takes the same approach immediately
below: chunk CVs are added only after the caller supplies at least 1 byte for
the following chunk.

\subsubsection{Chaining Value Stack}\label{sec:cvstack}

To help picture the role of the CV stack, Figure~\ref{fig:incrementaltrees}
shows a growing tree as chunk CVs are added incrementally. As just discussed
above, chunk CVs are added to this tree only after the caller has supplied at
least 1 byte for the following chunk, so we know that none of these chunks or
parent nodes is the root of the tree, and we do not need to worry about the
\flag{ROOT} flag yet.

\begin{figure}[h]
\centering
\begin{tikzpicture}

\node[draw,rectangle] (chunk1) at (0, 0) {$c_0$};

%%

\node[draw,rectangle,right=1.5cm of chunk1,dotted] (chunk2) {$c_0$};
\node[draw,rectangle,right=0.5cm of chunk2,dotted] (chunk3) {$c_1$};
\node[draw,rectangle,above=1cm of chunk2] (parent1) at ($(chunk2)!0.5!(chunk3)$) {$p_0$};
\draw[-latex] (chunk2) -- (parent1);
\draw[-latex] (chunk3) -- (parent1);

%%

\node[draw,rectangle,right=1.5cm of chunk3,dotted] (chunk2) {$c_0$};
\node[draw,rectangle,right=0.5cm of chunk2,dotted] (chunk3) {$c_1$};
\node[draw,rectangle,above=1cm of chunk2] (parent1) at ($(chunk2)!0.5!(chunk3)$) {$p_0$};
\node[draw,rectangle,right=0.5cm of chunk3] (chunk4) {$c_2$};
\draw[-latex] (chunk2) -- (parent1);
\draw[-latex] (chunk3) -- (parent1);

%%

\node[draw,rectangle,right=1.5cm of chunk4,dotted] (chunk2) {$c_0$};
\node[draw,rectangle,right=0.5cm of chunk2,dotted] (chunk3) {$c_1$};
\node[draw,rectangle,above=1cm of chunk2,dotted] (parent1) at ($(chunk2)!0.5!(chunk3)$) {$p_0$};
\node[draw,rectangle,right=0.5cm of chunk3,dotted] (chunk4) {$c_2$};
\node[draw,rectangle,right=0.5cm of chunk4,dotted] (chunk5) {$c_3$};
\node[draw,rectangle,above=1cm of chunk4,dotted] (parent2) at ($(chunk4)!0.5!(chunk5)$) {$p_1$};
\node[draw,rectangle,above=1cm of parent2] (parent3) at ($(parent1)!0.5!(parent2)$) {$p_2$};
\draw[-latex] (chunk2) -- (parent1);
\draw[-latex] (chunk3) -- (parent1);
\draw[-latex] (chunk4) -- (parent2);
\draw[-latex] (chunk5) -- (parent2);
\draw[-latex] (parent1) -- (parent3);
\draw[-latex] (parent2) -- (parent3);

\end{tikzpicture}
\caption{An incomplete tree growing incrementally from 1 to 4 chunks. Dotted
    boxes represent CVs that no longer need to be stored.}
\label{fig:incrementaltrees}
\end{figure}

When the first chunk CV ($c_0$) is added, it is alone. When the second chunk CV
($c_1$) is added, it completes a two-chunk subtree, and we can compute the
first parent CV ($p_0$). The third chunk CV ($c_2$) does not complete any
subtrees. Its final position in the tree structure will depend on whether it
gets a right sibling (see Figure~\ref{fig:fourchunks}), so we cannot create any
parent nodes for it yet. The fourth chunk CV ($c_3$) provides a right sibling
for $c_2$ and completes two subtrees, one of two chunks and one of four chunks.
First it merges with $c_2$ to compute $p_1$, then $p_1$ merges with $p_0$ to
compute $p_2$.

Note that once a subtree is complete, none of its child CVs will be used again,
and we do not need to store them any longer. These unneeded CVs are represented
by dotted boxes in Figure~\ref{fig:incrementaltrees}. Solid boxes represent CVs
that are still needed. These are what we store in the CV stack, ordered based
on when we computed them, with newer CVs on top.

Look through Figure~\ref{fig:incrementaltrees} again, this time paying
attention to the state of the CV stack at each step. When $c_0$ is added to the
tree, it is the only entry in the stack. When $c_1$ is added, we pop $c_0$ off
the stack to merge it, and $p_0$ becomes the only entry. When $c_2$ is added,
there are two entries in the stack, with $c_2$ on top because it is newer. When
$c_3$ is added, we perform two merges, first popping off $c_2$ and then popping
off $p_0$, and only $p_2$ remains in the stack.

Note how the stack order of $c_2$ and $p_0$ above match the order in which we
need to retrieve them for merging. By ordering CVs in the stack from newest at
the top to oldest at the bottom, we also implicitly order them from tree-right
to tree-left, and from the smallest subtree to the largest subtree. This
invariant means that the next CV we need to merge is always on top of the
stack.

The key question is then, when a new chunk CV is added to the tree, how many
subtrees does it complete? Which is to say, how many CVs should we pop off the
stack and merge with it, before pushing on the final result?

To answer this question, note another pattern. The entries in the CV stack
behave like the 1-bits in the binary representation of the total number of
chunks so far. For example with 3 chunks (\texttt{0b11}), there are two entries in the
stack, corresponding to the two 1-bits. With 4 chunks (\texttt{0b100}), only one entry
remains, corresponding to the single 1-bit.

To see why this pattern holds, note that any completed subtree represented by a
CV in the stack contains a number of chunks that is a power of 2. Furthermore,
each subtree is a distinct power of 2, because when we have two subtrees of the
same size, we always merge them. Thus, the subtrees in the stack correspond to
the distinct powers of 2 that sum up to the current total number of chunks.
This is equivalent to the 1-bits in the binary representation of that number.

In this sense, popping a CV off the stack to merge it is like flipping a 1-bit
to 0, and pushing the final result back onto the stack is like flipping a 0-bit
to 1. The fact that we do two merges when adding the fourth chunk, corresponds
to the fact that two 1-bits flip to 0 as the total changes from 3 to 4. In
other words, we do two merges when we add the fourth chunk, because there are
two trailing 1-bits in the number 3, and equivalently two trailing 0-bits in
the number 4.

This pattern leads to an algorithm for adding a new chunk CV to the tree: For
each trailing 0-bit in the new total number of chunks, pop a CV off the stack,
and merge it with the new CV we were about to add. Finally, push the fully
merged result onto the stack. Listing~\ref{listing:push_chunk_chaining_value}
shows this algorithm as it appears in the Rust reference implementation.

\begin{listing}[h]
\begin{minted}[fontsize=\footnotesize]{rust}
fn add_chunk_chaining_value(&mut self, mut new_cv: [u32; 8], mut total_chunks: u64) {
    // This chunk might complete some subtrees. For each completed subtree,
    // its left child will be the current top entry in the CV stack, and
    // its right child will be the current value of `new_cv`. Pop each left
    // child off the stack, merge it with `new_cv`, and overwrite `new_cv`
    // with the result. After all these merges, push the final value of
    // `new_cv` onto the stack. The number of completed subtrees is given
    // by the number of trailing 0-bits in the new total number of chunks.
    while total_chunks & 1 == 0 {
        new_cv = parent_cv(self.pop_stack(), new_cv, self.key, self.flags);
        total_chunks >>= 1;
    }
    self.push_stack(new_cv);
}
\end{minted}
\caption{The algorithm in the Rust reference implementation that manages the
    chaining value stack when a new chunk CV is added.}
\label{listing:push_chunk_chaining_value}
\end{listing}

Once we reach the end of the input, we can use a simpler algorithm to compute
the root hash. Finalize the CV of the current chunk, which may be incomplete or
empty. Then, merge that CV with every CV currently in the stack, regardless of
the number of chunks so far. Set the \flag{ROOT} flag for the last parent node
to be merged. Or if there are no CVs in the stack, and thus no parents to
merge, set the \flag{ROOT} flag for chunk finalization instead.

\subsection{Multi-threading}\label{sec:multithreading}

Most of the work of computing a BLAKE3 hash is compressing chunks. Each chunk
can be compressed independently, and one approach to multi-threading is to farm
out individual chunks or groups of chunks to tasks on a thread pool. In this
approach, a leader thread owns the chaining value stack (see
\S\ref{sec:cvstack}) and awaits a CV from each task in order.

This leader-workers approach has some inefficiencies. Spawning tasks and
creating channels usually requires heap allocation, which is a performance cost
that needs to be amortized over larger groups of chunks. At high degrees of
parallelism, managing the CV stack itself can become a bottleneck.

A recursive divide-and-conquer approach can be more efficient. The input is
split into left and right parts. As per the rules in \S\ref{sec:tree}, the left
part receives the largest power-of-2 number of chunks that leaves at least 1
byte for the right part. Each part then repeats this splitting step
recursively, until the parts are chunk-sized, and each chunk is compressed into
a CV. On the way back up the call stack, each pair of left and right child CVs
is compressed into a parent CV.

This recursive approach is good fit for a fork-join concurrency model, like
those provided by OpenMP, Cilk, and Rayon (Rust). Each left and right part
becomes a separate task, and a work-stealing runtime parallelizes those tasks
across however many threads are available. This can work without heap
allocation, because the runtime can make a fixed-size stack allocation at each
recursive call site.

The recursive approach is simplest when the entire input is available at once,
since no CV stack is needed in that case. In an incremental setting, a hybrid
approach is also possible. A large buffer of input can be compressed
recursively into a single subtree CV, and that CV can be pushed onto the CV
stack using the same algorithm as in \S\ref{sec:cvstack}. If each incremental
input is a fixed power-of-2 number of chunks in size (for example if all input
is copied into an internal buffer before compression), the algorithm works with
no modification. If each incremental input is instead a variable size (for
example if input from the caller is compressed directly without copying), the
implementation needs to carefully maintain the largest-to-smallest ordering
invariant of the CV stack. The size of each subtree being compressed must
evenly divide the total number of input bytes received so far, and to achieve
that the implementation might need to break up an incremental input into
smaller pieces. In the non-buffering case, the implementation must also take
care not to compress a parent node that could be the root, when it is unknown
whether more input is coming.

\subsection{SIMD}\label{sec:simd}

There are two approaches to using SIMD in a BLAKE3 implementation, and both are
important for high performance at different input lengths. The first approach
is to use 128-bit vectors to represent the 4-word rows of the state matrix. The
second approach is to use vectors of any size to represent words in multiple
states, which are compressed in parallel.

The first approach is similar to how SIMD is used in BLAKE2b or BLAKE2s, and it
is applicable to inputs of any length, particularly short inputs where the
second approach does not apply. The state $v_0 \ldots v_{15}$ is arranged into
four 128-bit vectors. The first vector contains the state words $v_0 \ldots
v_3$, the second vector contains the state words $v_4 \ldots v_7$, and so on.
Implementing the G function (Appendix~\ref{sec:roundfn}) with vector
instructions thus mixes all four columns of the state matrix in parallel. A
diagonalization step then rotates the words within each row so that each
diagonal now lies along a column, and the vectorized G function is repeated to
mix diagonals. Finally the state is undiagonalized, to prepare for the column
step of the following round.

The second approach is similar to how SIMD is used in BLAKE2bp or BLAKE2sp. In
this approach, multiple chunks are compressed in parallel, and each vector
contains one word from the state matrix of each chunk. That is, the first
vector contains the $v_0$ word from each state, the second vector contains the
$v_1$ word from each state, and so on, using 16 vectors in total. The width of
the vectors determines the number of chunks, so for example 128-bit vectors
compress 4 chunks in parallel, and 256-bit vectors compress 8 chunks in
parallel. Here the G function operates on one column or diagonal at a time, but
across all of the states, and no diagonalization step is required. When enough
input is available, this approach is much more efficient than the first
approach. It also scales to wider instruction sets like AVX2 and AVX-512.

The second approach can be integrated with the CV stack algorithm from
\S\ref{sec:cvstack} by computing multiple chunk CVs in parallel and then
pushing each of them into the CV stack one at a time. It can also be combined
with either of the multi-threading strategies from \S\ref{sec:multithreading}.
Rather than having each task or recursive leaf compress one chunk at a time,
each can compress multiple chunks in parallel.

\subsection{Memory Requirements}\label{sec:memory}

BLAKE3 has a larger memory requirement than BLAKE2, because of the chaining
value stack described in \S\ref{sec:cvstack}. An incremental implementation
needs space for a 32-byte chaining value for every level of the tree below the
root. The maximum input size is $2^{64}-1$ bytes, and the chunk size is
$2^{10}$ bytes, giving a maximum tree depth of $64 - 10 = 54$. The CV stack
thus requires $54 \cdot 32 = 1728$ bytes. The chunk state
(cf.~\S\ref{sec:chunkstate}) also requires at least 104 bytes for the chaining
value, the message block, and the chunk counter. The size of the reference
implementation is 1880 bytes on the call stack.

For comparison, BLAKE2s has a memory footprint similar to the BLAKE3 chunk
state alone, at least 104 bytes. BLAKE2b has twice the chaining value size and
block size, requiring at least 200 bytes. And the parallel modes BLAKE2bp and
BLAKE2sp both require at least 776 bytes.

Space-constrained implementations of BLAKE3 can save space by restricting the
maximum input size. For example, the maximum size of an IPv6 ``jumbogram'' is
$2^{32}-1$ bytes, or just under 4~GiB. At this size, the tree depth is 22 and
the CV stack is $22 \cdot 32 = 704$ bytes. Similarly, the maximum size of a TLS
record is $2^{14}$ bytes, or exactly 16~KiB. At this size, the tree depth is 4
and the CV stack is $4 \cdot 32 = 128$ bytes.

\section{Applications}\label{sec:applications}

As a general-purpose hash function, BLAKE3 is suitable whenever a
collision-resistant or preimage-resistant hash function is needed to map
some arbitrary-size input to a fixed-length output.
BLAKE3 further supports keyed modes---in order to be used as a pseudorandom
function, MAC, or key derivation function---as well as streaming and
incremental processing features.

\subsection{Pseudorandom Function and MAC}\label{sec:mac}

Like BLAKE2, BLAKE3 provides a keyed mode, \flag{keyed_hash}. This removes the
need for a separate construction like HMAC. The \flag{keyed_hash} mode is also
more efficient than keyed BLAKE2 or HMAC for short messages. BLAKE2 requires an
extra compression for the key block, and HMAC requires three extra
compressions. The \flag{keyed_hash} mode in BLAKE3 does not require any extra
compressions.

\subsection{Key Derivation}\label{sec:kdf}

BLAKE3 provides a key derivation mode, \flag{derive_key}. This mode accepts a
context string of any length and key material of any length, and it returns a
derived key of any length. The context string should be hardcoded, globally
unique, and application-specific. A good default format for context strings is
\texttt{"[application] [commit timestamp] [purpose]"}, e.g.,
\texttt{"example.com 2019-12-25 16:18:03 session tokens v1"}. This mode should
not be used with a variable context string, unless that variable is itself a
context parameter from a higher level API with the same documented security
requirement.

The purpose of this strict requirement is to ensure that there is no way for an
attacker in any scenario to cause two different applications or components to
inadvertently use the same context string. There should be no dependency
whatsoever between user input and a context string used with \flag{derive_key}.

Given this requirement, we then make an explicit exception to an otherwise
universal rule of practical cryptography: Applications may use
\flag{derive_key} with key material that is already in use with other
algorithms. This includes other hash functions, ciphers, and abstractions like
HMAC and HKDF. The only limitation in practice is algorithms that forbid key
reuse entirely, like a one-time pad. (For other theoretical limitations, see
\S\ref{sec:reusedkeys}.)

Normally, using the same key with multiple algorithms is forbidden. Many
algorithms are built from the same primitives, and it can be difficult to know
when two algorithms might have related output, or when one algorithm might
publish a value that another algorithm considers secret. One especially
relevant example of this problem is HMAC and HKDF themselves. Although these
algorithms have entirely different purposes, HKDF is defined in terms of HMAC.
In fact, in the ``expand only'' mode of HKDF, its output is equivalent to a
single call to HMAC. If an application used the same key with HMAC and with
HKDF, and an HMAC input happened to collide with an HKDF context string, this
interaction could ruin the application's security.

The standard advice for such an application is to use its original key only
with a key derivation function, and to derive independent subkeys for other
algorithms. This is good and practical advice in many cases. But it may not be
practical when an application or a protocol evolves over time. If a key was
originally used only with HMAC, and perhaps years later a second use case
arises, the developers may have painted themselves into a corner. Backwards
compatibility might prevent them from using a derived subkey with HMAC. In
cases like this, the standard advice amounts to either building a time machine,
or over-engineering applications and protocols to derive subkeys at every step,
in anticipation of unknown future use cases.

This is why it is valuable for \flag{derive_key} to make an exception to the
rule against mixing algorithms. It makes key derivation a practical option not
only for new designs, but also for existing applications that are adding new
features. In exchange, it is the responsibility of each caller to provide (or
for abstractions, to require) a hardcoded, globally unique,
application-specific context string, which can never be made to collide with
any other.

The \flag{derive_key} mode is intended to replace the BLAKE2 personalization
parameter for most of its use cases. Key derivation can encourage better
security than personalization with a shared key, by cryptographically isolating
different components of an application from one another. This limits the damage
that one component can cause by accidentally leaking its key.

\subsection{Stateful Hash Object}\label{sec:sho}

To abstract away various uses of the hash function in the Noise protocol framework, the notion of \emph{stateful hash objects} (SHO)~\cite{Perrin19} has been suggested. An SHO is an interface with 3 methods:
\begin{itemize}
  \item \texttt{init(label)}: Initializes the internal state using a custom domain-separation label.
  \item \texttt{absorb(data)}: Hashes an arbitrary number of bytes into the state.
  \item \texttt{squeeze(length)}: Produces an arbitrary amount of pseudorandom output based on the current state.
\end{itemize}
Due to its zero-overhead keying and variable-length output, it is straightforward to adapt BLAKE3 into an SHO. The internal state at the time $t$ is represented by a $256$-bit key $k_t$:
\begin{itemize}
  \item The initial key, $k_0$, is obtained as $k_0 = \text{\flag{hash}}(label)$;
  \item \texttt{absorb(data)} updates the key: $k_{i+1} = \text{\flag{keyed_hash}}(k_i, data)$;
  \item \texttt{squeeze(length)} extends the output of the previous call to \flag{hash} or \flag{keyed_hash}.\footnote{Extending the output of the last absorb entails preserving the last up to $64$ bytes of input data in memory for possibly a long time; in such cases it might be desirable to call \texttt{absorb(empty)} after each absorb call.}.
\end{itemize}

\subsection{Verified Streaming}\label{sec:verifiedstreaming}

Because BLAKE3 is a tree hash, it supports new use cases that serial hash
functions do not. One such use case is verified streaming. Consider a video
streaming application that fetches video files from an untrusted host. The
application knows the hash of the file it wants, and it needs to verify that
the video data it receives matches that hash. With a serial hash function,
nothing can be verified until the application has downloaded the entire file.
But with BLAKE3, verified streaming is possible. The application can verify and
play individual chunks of video as soon as they arrive.

To verify an individual chunk without re-hashing the entire file, we verify
each parent node on the path from the root to that chunk. For example, suppose
the file is composed of four chunks, like the four-chunk tree in
Figure~\ref{fig:fourchunks}. To verify the first chunk, we start by fetching
the root node. (Specifically, we fetch its message bytes, the concatenated
chaining values of its children.) We compute the root node's CV as per
\S\ref{sec:parent} and confirm that it matches the 32-byte BLAKE3 hash of the
entire file. Then, we fetch the root node's left child, which is the first
chunk's parent. We compute that node's CV and confirm that it matches the first
32-bytes of the root node. Finally, we fetch the first chunk itself. We compute
its CV as per \S\ref{sec:chunk} and verify that it matches the first 32-bytes
of its parent. This verifies that the first chunk is authentic, and we can pass
it along to application code.

To continue streaming, we can immediately fetch the second chunk. It shares the
parent node of the first chunk, and its CV should match the second 32 bytes of
that parent node. For the third chunk, we need to fetch its parent. The CV of
that parent node should match the second 32 bytes of the root node, and then
the CV of the third chunk should match the first 32 bytes of its parent. Note
that whenever we fetch a parent node, we immediately use its first 32 bytes to
check its left child's CV, and then we store its second 32 bytes to check its
right child in the future. We can represent this with a stack of expected CVs.
We push CVs onto this stack and pop them off, as we would with node pointers in
a depth-first traversal of a binary tree. Note that this is different from the
``CV stack'' used for incremental hashing in \S\ref{sec:cvstack}; that stack
holds CVs we have computed in the past, while this stack holds CVs we expect to
compute in the future, and we manage them with different algorithms.

Observe that if the application eventually verifies the entire file, it will
have fetched all the nodes of the tree in pre-order. This suggests a simple
wire format for a streaming protocol: The host can concatenate all the parent
nodes and chunks together in pre-order and serve the concatenated bytes as a
single stream. If the client application knows the length of the file in
advance, it does not need any other information to parse the stream and verify
each node. In this format, the space overhead of the parent nodes approaches
two CVs per chunk, or 6.25\%.

If the client does not know the length of the file in advance, it may receive
the length from the host, either separately or at the front of the stream. In
this case, the length is untrusted. If the host reports an incorrect length,
the effect will be that at least the final chunk will fail verification. For
this reason, if an application reads the file sequentially, it does not need to
explicitly verify the length. An important edge case is that, if the reported
length is 0, the file consists of a single empty chunk, and the implementation
must not forget to verify that the file hash matches the empty chunk's CV. One
way to make this mistake is for the implementation to return end-of-file as
soon as the current read position equals the expected length, verifying nothing
in the zero-length case.

On the other hand, if an application implements seeking, length verification is
required. Seeking past the reported end-of-file, or performing an EOF-relative
seek, would mean trusting an unverified length. In these cases, the
implementation must first verify the length by seeking to and verifying the
final chunk. If that succeeds, the length is authentic, and the implementation
can proceed with the caller's requested seek.

This verified streaming protocol has been implemented by the
\href{https://github.com/oconnor663/bao}{Bao} tool. It is conceptually similar
to the existing
\href{https://adc.sourceforge.io/draft-jchapweske-thex-02.html}{Tree Hash
Exchange} format, and to the
\href{https://www.bittorrent.org/beps/bep_0030.html}{BEP~30} extension of the
BitTorrent protocol. However, note that neither of those prevents length
extension, and the latter (if extracted from the BitTorrent protocol) does not
provide second-preimage resistance~\cite[\S8.5]{DBLP:journals/tosc/DaemenMA18}.

\subsection{Incremental Updates}\label{sec:incrementalupdate}

Another new use case supported by BLAKE3 is incrementally updating the root
hash when an input is edited in place. A serial hash function can efficiently
append new bytes to the end of an input, but editing bytes at the beginning or
in the middle requires re-hashing everything to the right of the edit. With
BLAKE3, an application can edit bytes anywhere in its input and re-hash just
the modified chunks and their transitive parent nodes.

For example, consider an input composed of four chunks, again like the
four-chunk tree in Figure~\ref{fig:fourchunks}. Suppose we overwrite some bytes
in the second chunk. To update the root hash, we first compute the second
chunk's new CV as per \S\ref{sec:chunk}. Then we recompute the CV for the
second chunk's parent as per \S\ref{sec:parent}. Note that because the first
chunk is unchanged, the first 32 message bytes for this parent node are
unchanged also, and we do not need to re-read the first chunk. Finally, we
recompute the CV of the root node, and we similarly do not need to re-read
anything from the right half of the tree.

Note that this strategy cannot efficiently insert or remove bytes from the
middle of an input. That would change the position of all the bytes to the
right of the edit and force re-hashing. This is similar to the constraints of a
typical filesystem, where appends and in-place edits are efficient, but
insertions and removals within a file are either slow or unsupported.

\section{Rationales}\label{sec:rationales}

This section goes into more detail about some of the performance tradeoffs and
security considerations in the BLAKE3 design.

\subsection{Chunk Size}\label{sec:chunksize}

BLAKE3 uses 1~KiB chunks. The chunk size is a tradeoff between the peak
throughput for long inputs, which benefits from larger chunks, and the degree
of parallelism for medium-length inputs, which benefits from shorter chunks.

The benefit of a larger chunk size is that the tree contains fewer parent
nodes, so the overhead cost of compressing them is lower. The number of parent
nodes is equal to the number of chunks minus one, so doubling the chunk size
cuts the number of parent nodes in half. However, note that parent nodes at the
same level of the tree can be compressed in parallel, so having twice as many
parent nodes does not necessarily mean that compressing them takes twice as
long.

The benefit of a smaller chunk size is a higher potential degree of
parallelism. This is irrelevant for very long inputs, where all the SIMD lanes
and possibly CPU cores of the machine will be occupied regardless. But at
medium input lengths, this has a huge impact on performance. Consider an 8~KiB
input. With a chunk size of 1~KiB, this input can occupy 8 SIMD lanes, which
lets the implementation use AVX2 vector instructions on modern x86 processors.
But if we increased the chunk size to 2 KiB, the same input would only occupy 4
SIMD lanes, and the high throughput of AVX2 would be left on the table.

Thus we want to pick the smallest chunks size we can to maximize medium-length
parallelism, without incurring ``excessive'' overhead from parent nodes. The
point of comparison here is the peak throughput for a very large chunk size,
like 64~KiB, where parent node overhead becomes small enough to be negligible.
In our measurements, we find that implementations with 1~KiB chunks can reach
at least 90\% of that peak throughput.

There are other minor performance considerations to be aware of. A larger chunk
size slightly reduces the space requirement of the CV stack as described in
\S\ref{sec:memory}. Also, verified streaming requires buffering chunks (see
\S\ref{sec:verifiedstreaming}), and incremental updates require re-hashing
chunks (see \S\ref{sec:incrementalupdate}), so both of those use cases benefit
from a smaller chunk size.

\subsection{Word Size}\label{sec:wordsize}

BLAKE3 uses 32-bit words. The performance tradeoffs in the choice between
32-bit and 64-bit words are complex, and they depend on both the platform and
the size of the input.

It is common advice that SHA\=/512 (because it uses 64-bit words) is faster on
64-bit platforms than SHA\=/256 (because it uses 32-bit words). A similar
effect applies to BLAKE2b and BLAKE2s. However, this effect does not apply to
algorithms that incorporate more SIMD parallelism. BLAKE2sp has higher peak
throughput than BLAKE2bp on x86-64. This is because the SIMD instructions that
operate on 8 lanes of 32-bit words have the same throughput as those that
operate on 4 lanes of 64-bit words. The key factor for exploiting SIMD
performance is not word size but vector size, and both of those cases occupy a
256-bit vector. (The remaining effect making BLAKE2sp faster is that its
smaller state requires fewer rounds of compression.)

For that reason, peak throughput on x86-64 is largely independent of the word
size. The performance differences that remain are restricted to shorter input
lengths: 32-bit words perform better for lengths less than one block, while
64-bit words perform better for lengths between one block and one chunk,
comparable to BLAKE2s and BLAKE2b. These effects are minor, and in any case
short input performance is often dominated by other sources of overhead in
typical applications.

Instead, the decisive advantage of 32-bit words over 64-bit words is
performance on smaller architectures and embedded systems. BLAKE3 is designed
be a single general-purpose hash function, suitable for users of both BLAKE2b
and BLAKE2s. On 32-bit platforms like the ARM1176, where BLAKE2s does
especially well, a 64-bit hash function could be a performance regression.
Choosing 32-bit words is a substantial benefit for these platforms, with little
or no downside for x86-64. Cross-platform flexibility is also important for
protocol designers, who have to consider the performance of their designs
across a wide range of hardware, and who are becoming increasingly skeptical of
designs supporting multiple algorithms~\cite{WG}.

\subsection{Compression Function Feed Forward}\label{sec:feedforward}

The output step of the compression function (see \S\ref{sec:compression}) mixes
the first and second halves of the state, and then feeds forward the input CV
into the second half. The feed-forward step allows us to use all 16 words of
the state as output, doubling the output rate of the extendable output feature
(see \S\ref{sec:extendable}). If we exposed all 16 state words without the
feed-forward, the compression function would be reversible when the message
bytes were known.

When the full output is used, mixing the first and second halves of the state
is redundant. However, when the truncated output is used, mixing the halves
makes the output use all the state words computed during the last round. The
default 32-byte output size is a truncated output, and truncated outputs are
also used for chaining values in the interior of the tree.

Putting the feed-forward in the second half of the state allows an
implementation that does not support extendable output to skip the feed-forward
entirely.

\subsection{Chunk Counter}\label{sec:chunkcounter}

The main job of the compression function's 64-bit counter parameter, $t$, is to
support extendable output. By incrementing $t$ (see \S\ref{sec:extendable}),
the caller can extract as much output as needed, without imposing any extra
steps on the default-length case.

However, we also use the $t$ parameter during the input phase, to indicate the
chunk number (see \S\ref{sec:chunk}). This means that each chunk has a distinct
CV with high probability, even if two chunks have the same input bytes. This is
not strictly necessary for security, but it discourages a class of dangerous
optimizations.

Consider an application that hashes sparse files, which are mostly filled with
zeros. The majority of this application's input chunks could be the all-zero
chunk. This application might try to compute the CV of the zero chunk only
once, and then check each chunk before compressing it to see whether it can use
that precomputed CV. This check is cheap, and for this application it could be
a big speedup. But this check is also dangerous, because it is not
constant-time. The runtime of the hash function would leak information about
its input.

If tricks like this were possible, an unsafe implementation would inevitably
find its way into some privacy-sensitive use case. By distinguishing each
chunk, BLAKE3 deliberately sabotages these tricks, in the hopes of keeping
every implementation constant-time.

\subsection{Tree Fanout}\label{sec:treefanout}

BLAKE3 uses a binary tree structure, which is to say a fanout of 2. A binary
tree is simpler to implement than a wider tree, and its memory requirement is
also a local minimum.

The CV stack algorithm described in \S\ref{sec:cvstack} is important for
simplifying a BLAKE3 implementation. It also saves space, because the
implementation does not need to store size or level metadata along with each
CV. Consider how this algorithm would change, if BLAKE3 used a fanout of 4.
Instead of ``count the trailing 0-bits, and pop that many CVs'' it would be
``count the trailing 0-0-bit pairs, and pop that many CV triplets.'' That can
still be done with a bit mask, but programmers would need to reason about
quaternary numbers to understand what was going on. Other complexities would
crop up as well: parent nodes would no longer be a fixed size, and new
ambiguities in the tree layout would require a third rule in \S\ref{sec:tree}
along the lines of ``all non-rightmost sibling subtrees have the same number of
chunks.''

A fanout of 4 would also would also increase the memory requirement of the CV
stack. Instead of storing one CV per level of the tree, we would need to store
three. The tree would be half as tall, but ultimately the CV stack would be
larger by a factor of $3/2$. In general, the size of the CV stack for a fanout
$F$ is proportional to $(F-1)/\log_2(F)$, which increases as $F$ increases.

A different approach to the CV stack is possible, which could save space at
larger fanouts. Rather than storing CV bytes directly, the implementation could
store an incremental state for each incomplete parent node, similar to the
incremental chunk state. An incremental state requires a 32-byte CV and a
64-byte buffer, so at fanout 4 it would be no better than storing 3 CVs
directly. (Perhaps the 64-byte buffer could be reduced to 32 bytes in some
cases, depending on the details of parent node finalization, and at the cost of
more complexity.) But at higher fanouts, the memory requirement would be
proportional to $1/\log_2(F)$. At fanout 16 and above, this approach could save
space relative to the binary tree.

However, recall from \S\ref{sec:memory} that space-constrained implementations
of BLAKE3 can reduce the size of the CV stack by restricting their maximum
input length. We expect that the majority of space-constrained applications
already have strict limits on the size of their inputs. Indeed it is hard to
imagine how an application without 2 KB of memory to spare could even get its
hands on a gigabyte of input. Realistically, a large fanout would only benefit
a small fringe of space-constrained applications, but the added complexity
would hurt everyone.

As a special case, we could also have chosen an infinite fanout, a single
parent node with an unlimited number of leaves. This approach does not require
a CV stack at all. Instead, the implementation only needs to manage the
incremental state of the current chunk and the incremental state of the single
parent node. Notably, this is the approach taken by
KangarooTwelve~\cite{DBLP:conf/acns/BertoniDPAKV18}. It has large benefits for
simplicity and space efficiency, but also several downsides. Parent blocks
cannot be compressed in parallel, requiring a larger chunk size to keep
overhead low (see \S\ref{sec:chunksize}). Multi-threading is more complex and
generally requires heap allocation, because the input cannot be split in half
recursively (see \S\ref{sec:multithreading}). And incremental use cases like
verified streaming (see \S\ref{sec:verifiedstreaming}) are no longer practical.

As a totally different approach, we could also have chosen a fixed-interleaved
tree layout similar to that of BLAKE2bp and BLAKE2sp. For example, BLAKE2sp
divides its input into 8 interleaved pieces, with blocks 0, 8, 16... forming
the first piece, blocks 1, 9, 17... forming the second piece, and so on. The 8
resulting CVs are then concatenated into a single parent node. The upsides of
this approach are that full parallelism can begin at relatively short input
lengths, 512 bytes in the case of BLAKE2bp and BLAKE2sp, and that the memory
requirement is lower than that of the binary tree. The downsides of this
approach are that the degree of parallelism is fixed, and that performance is
poor for shorter inputs (see Figures~\ref{fig:avx512} and~\ref{fig:rpizero}).

\subsection{Domain Flags}\label{sec:domainflags}

The \flag{ROOT} and \flag{CHUNK_START} domain flags are strictly necessary for
security. The \flag{ROOT} flag separates the root node from all other nodes,
and it separates the final block from all other blocks in a root chunk, both of
which prevent length extension. The \flag{CHUNK_START} flag separates chunks
from parent nodes, and it separates an IV provided by the caller (the key in
the keyed hashing mode) from block CVs computed within a chunk, both of which
prevent collisions.

The \flag{PARENT} and \flag{CHUNK_END} domain flags are not strictly necessary,
but we include them to be conservative. As with \flag{CHUNK_START}, the
\flag{PARENT} flag prevents a collision when the caller-provided IV bytes match
a block CV extracted from some other input. However, the parent's message bytes
will also be different from the block's with high probability in that case. The
benefit of the \flag{PARENT} flag is rather that we get separation without
needing to reason about the message bytes.

The \flag{CHUNK_END} flag prevents length extensions for all chunks, beyond
just a root chunk. However, other chunk CVs are not usually published. In
incremental use cases like verified streaming (see
\S\ref{sec:verifiedstreaming}), chunk CVs may be published, but only when the
input is also published. In a keyed incremental use case, the keyed parent
nodes also prevent an attacker from making any use of length-extended keyed
chunk CVs. Nonetheless, we include the \flag{CHUNK_END} flag, because letting
an attacker length-extend chunk CVs is an unnecessary risk that might impact
unknown future applications.

\subsection{Key Derivation from Reused Key Material}\label{sec:reusedkeys}

As described in \S\ref{sec:kdf}, the \flag{derive_key} mode is intended to be
safe to use even with keys that are already in use with other algorithms. It is
impossible to guarantee such a property in general, however. As noted in that
section, some algorithms explicitly forbid key reuse, even within the same
algorithm. In the case of a one-time pad, if the message happens to be known or
predictable, some or all of the key is revealed to an eavesdropper. No key
derivation function can extract useful security from key material that is not
secret.

There are some algorithms that impose a one-time-use restriction on their key,
but that still keep the key secret as long as that restriction is followed. The
\flag{crypto_onetimeauth} function in the NaCl and libsodium libraries is one
such example. It might be safe to use the same key with
\flag{crypto_onetimeauth} (once) and with \flag{derive_key} (any number of
times). However, precisely because of its one-time-use restriction,
\flag{crypto_onetimeauth} is almost always used with derived subkeys anyway.
Furthermore, this sort of algorithm-specific reasoning is exactly what
\flag{derive_key} is designed to avoid. It is simpler and safer to forbid
\flag{derive_key} from sharing key material with this entire class of
one-time-use algorithms.

Another limitation on key reuse is applications that violate the security
requirement of the \flag{derive_key} context string, namely that it should be
hardcoded, globally unique, and application-specific. If one component of an
application feeds arbitrary user input into the context string, it is not safe
for any other component to share key material with it. The context string no
longer provides domain separation. This is the basic problem of sharing key
material, and the reason that deriving separate subkeys is preferable wherever
possible. When two different components share a key, each has to assume that
the other will not violate the security requirements of \flag{derive_key}, just
as each has to assume that the other will not leak the key.

Another more theoretical limitation for key material reuse could be future
algorithms that are internally identical to BLAKE3, but with different
domain-separation conventions. A modified algorithm might sabotage domain
separation entirely, for example, by exposing the $d$ parameter of the
compression function to arbitrary user input. We will never publish such an
algorithm. The common practice when designing a closely related algorithm is to
change the IV constants, so that the new output is independent. This was the
approach used to derive SHA\=/512/256 from SHA\=/512. The same approach could work
for a hypothetical derivative of BLAKE3 by changing the $IV_0 \ldots IV_3$
constants, which are used for compression function setup in all modes.

% TODO: Let me know what you guys think about having an Acknowledgments section.
\section*{Acknowledgments}\label{sec:acknowledgments}

The development of BLAKE3 was sponsored by
\href{https://electriccoin.co/}{Electric Coin Company} and
\href{https://teserakt.io}{Teserakt}. Thanks to Liming Luo for many discussions
about the tree layout.

%\nocite{*}
\bibliographystyle{plainurl}
\bibliography{blake3}

\begin{appendices}

\section{IV Constants}\label{sec:ivconstants}

    The constants $\IV_0 \ldots \IV_7$ used by the compression function are the
    same as in BLAKE2s. They are:
\begin{align*}
    \IV_0 &= \text{\texttt{0x6a09e667}} &
    \IV_1 &= \text{\texttt{0xbb67ae85}} \\
    \IV_2 &= \text{\texttt{0x3c6ef372}} &
    \IV_3 &= \text{\texttt{0xa54ff53a}} \\
    \IV_4 &= \text{\texttt{0x510e527f}} &
    \IV_5 &= \text{\texttt{0x9b05688c}} \\
    \IV_6 &= \text{\texttt{0x1f83d9ab}} &
    \IV_7 &= \text{\texttt{0x5be0cd19}}
\end{align*}

\section{Round Function}\label{sec:roundfn}

    The compression function transforms the internal state $v_{0} \ldots
    v_{15}$ through a sequence of 7 rounds. The round function is the same as
    in BLAKE2s. A round does:
\begin{align*}
    \GG_{0}&(v_{0}, v_{4}, v_{8}, v_{12}) &
    \GG_{1}&(v_{1}, v_{5}, v_{9}, v_{13}) &
    \GG_{2}&(v_{2}, v_{6}, v_{10}, v_{14}) &
    \GG_{3}&(v_{3}, v_{7}, v_{11}, v_{15}) \\
    \GG_{4}&(v_{0}, v_{5}, v_{10}, v_{15}) &
    \GG_{5}&(v_{1}, v_{6}, v_{11}, v_{12}) &
    \GG_{6}&(v_{2}, v_{7}, v_{8}, v_{13}) &
    \GG_{7}&(v_{3}, v_{4}, v_{9}, v_{14})
\end{align*}

    That is, a round applies a G function to each of the columns in parallel,
    and then to each of the diagonals in parallel. $\GG_i(a, b, c, d)$ is
    defined as follows. $\oplus$~denotes bitwise exclusive-or, $\ggg$~denotes
    bitwise right rotation, and $m_{\sigma_r(x)}$ is the message word whose
    index is the $x$\textsuperscript{th} entry in the message schedule for
    round $r$:
\begin{align*}
    a \ & \leftarrow \ a + b + m_{\sigma_r(2i)} \\
    d \ & \leftarrow \ (d \oplus a) \ggg 16 \\
    c \ & \leftarrow \ c + d \\
    b \ & \leftarrow \ (b \oplus c) \ggg 12 \\
    a \ & \leftarrow \ a + b + m_{\sigma_r(2i+1)} \\
    d \ & \leftarrow \ (d \oplus a) \ggg 8 \\
    c \ & \leftarrow \ c + d \\
    b \ & \leftarrow \ (b \oplus c) \ggg 7
\end{align*}

    The message schedules $\sigma_r$ are:

\begin{center}
\begin{tabular}{ | r | r r r r r r r r r r r r r r r r | }
    \hline
    $\sigma_0$ & 0 & 1 & 2 & 3 & 4 & 5 & 6 & 7 & 8 & 9 & 10 & 11 & 12 & 13 & 14 & 15 \\
    $\sigma_1$ & 14 & 10 & 4 & 8 & 9 & 15 & 13 & 6 & 1 & 12 & 0 & 2 & 11 & 7 & 5 & 3 \\
    $\sigma_2$ & 11 & 8 & 12 & 0 & 5 & 2 & 15 & 13 & 10 & 14 & 3 & 6 & 7 & 1 & 9 & 4 \\
    $\sigma_3$ & 7 & 9 & 3 & 1 & 13 & 12 & 11 & 14 & 2 & 6 & 5 & 10 & 4 & 0 & 15 & 8 \\
    $\sigma_4$ & 9 & 0 & 5 & 7 & 2 & 4 & 10 & 15 & 14 & 1 & 11 & 12 & 6 & 8 & 3 & 13 \\
    $\sigma_5$ & 2 & 12 & 6 & 10 & 0 & 11 & 8 & 3 & 4 & 13 & 7 & 5 & 15 & 14 & 1 & 9 \\
    $\sigma_6$ & 12 & 5 & 1 & 15 & 14 & 13 & 4 & 10 & 0 & 7 & 6 & 3 & 9 & 2 & 8 & 11 \\
    \hline
\end{tabular}
\end{center}

\end{appendices}

\end{document}
